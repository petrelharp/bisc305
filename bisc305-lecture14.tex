
\lecture[14]{Statistical significance}{lecture-text}

\subtitle{and hypothesis testing principles}

\date{22 October 2013}


\begin{document}

\begin{frame}
  \maketitle
\end{frame}


\section*{Introduction}

\begin{frame}{Last time} 
  \begin{enumerate}
    \item Experimental vs observational studies
    \item Correlation is not causation
    \item Confounding factors
    \item the One-sided $t$-test
  \end{enumerate}
\end{frame}


\begin{frame}\frametitle<presentation>{Outline}
  \begin{quote}
    ``The statistical analysis should aid the researcher by helping to clarify whatever
    message is contained in the data.''
  \end{quote}
  \tableofcontents
\end{frame}


%%%%%% %%%%%%
\section{Significant or important?}

\subsection{Unpacking the terminology}

%%%%%%
\begin{frame}{Statistical significance}

    \begin{quote}
        the difference was \only<2->{\alert{statistically} }significant \\
        \only<2->{\alert{(with $P<.05$)}}
    \end{quote}

    \uncover<3->{
    \structure{means}
    \begin{quote}
        we found good evidence that the difference in sample means was not caused by chance variation alone
    \end{quote}
    }

    \uncover<4->{
    \structure{or more precisely}
    \begin{quote}
        if the null hypothesis was true, we'd expect to see at least this large a difference no more than 5\% of the time
    \end{quote}
    }

\end{frame}

%%%%%%
\begin{frame}{Lack of statistical signficance}

    \begin{quote}
        the difference was \alert{not} significant (with $P>0.5$)
    \end{quote}

    \uncover<2->{
    \structure{means}
    \begin{quote}
        there was not sufficient evidence that the observed difference was due to anything other than chance variation
    \end{quote}
    }

    \uncover<3->{
    \structure{or more precisely}
    \begin{quote}
        if the null hypothesis was true, we'd expect to see at least this large a difference \alert{at least} 5\% of the time
    \end{quote}
    }

\end{frame}

%%%%%%
\begin{frame}{Example}

  Lactate dehydrogenase levels:
  \begin{center}
    \begin{tabular}{crr}
       & Males & Females \\
       \hline
       $n$ & 270 & 264 \\
       $\bar y$ & 60 & 57 \\
       $s$ & 11 & 10
     \end{tabular}

   \vspace{2em}

   \uncover<2->{$t_s=3.3$ and $P=0.001$}
   \end{center}

\end{frame}


%%%%%%
\begin{frame}{Example}

  Body weight:
  \begin{center}
    \begin{tabular}{crr}
       & Males & Females \\
       \hline
       $n$ & 2 & 2 \\
       $\bar y$ & 175 & 143 \\
       $s$ & 35 & 34
     \end{tabular}

   \vspace{2em}

   \uncover<2->{$t_s=0.93$ and $P=0.45$}
   \end{center}

\end{frame}


%%%%%%
\begin{frame}{Interpret:}

  \begin{quote}
    ``Among 9 independent species pairs for which both pelvic bones and rib bones could be analyzed, pelvic bone shape was positively correlated with mating ecology ($t_s=2.7$, $P=0.008$), a pattern not observed for ribs ($t_s=0.2$, $P=0.40$).''
  \end{quote}
  \flushright { \small Dines et al 2014 }

\end{frame}



%%%%%% %%%%%% %%%%%% %%%%%%
\section{Describing real-world importance}


%%%%%%
\begin{frame}{It's significant, but is it important?}


    \begin{block}{the difference in means}
        We've observed a difference between two populations:
        \[  \bar y_1 - \bar y_2 \]
    \end{block}

    \ldots and got a small $P$-value.  So what?

    \vspace{2em}

    \structure{Ask yourself:} What do we want to know?\\
    How important is the observed difference?


\end{frame}


%%%%%% %%%%%%
\subsection{The effect size}

%%%%%%
\begin{frame}{Effect size}

    \begin{block}<1->{the $t$-statistic}
        \begin{align*}
            t_s = \frac{\bar y_1 - \bar y_2}{ \SE_{\bar Y_1 - \bar Y_2} } 
            = \frac{ \text{(actual difference)} }{ \text{(magnitude of sampling noise)} }
        \end{align*}
        is big if the observed difference is large compared to the estimated size of the \alert{estimation error}.
    \end{block}

    \begin{block}<2->{the effect size}
        If both populations have the same SD, $\sigma$,
        \begin{align*}
            \text{(Effect size)} = \frac{\bar y_1 - \bar y_2}{ \sigma }
            = \frac{ \text{(actual difference)} }{ \text{(magnitude of within-population variation)} }
        \end{align*}
        is big if the observed difference is large compared to \alert{within-population variation}.
    \end{block}

\end{frame}

%%%%%%
\begin{frame}{Examples}

  The \alert{effect size} is a good descriptive statistic.

  \vspace{2em}

  Examples:
  \begin{itemize}
    \item Lactate dehydrogenase levels:
    \item Body weight:
  \end{itemize}

\end{frame}


%%%%%% %%%%%%
\subsection{Using confidence intervals}

%%%%%%
\begin{frame}{Significantly unimportant?}

    The \alert{confidence interval} describes where we are pretty confident the true value lies.

    \begin{itemize}

    \item So if we think an effect smaller than $x$ is \alert{unimportant}
      and $x$ is outside the confidence interval,

    \item then we have good evidence the true effect is unimportant.
      (inconsequential, insignificant, \ldots?)

    \end{itemize}

    \vspace{2em}

    Examples:
    \begin{itemize}
      \item Lactate dehydrogenase levels:
      \item Body weight:
    \end{itemize}

\end{frame}

%%%%%%
\begin{frame}{Interpret:}

  \begin{quote}
    ``Our data also suggest a positive association of cigarette smoking and new HPV detection (OR, 3.4; 95\% CI, 0.4--26.3); {however, because 87\% of study participants were smokers, we had limited power to detect significant effects.}''
  \end{quote}

  \vspace{2em}

  OR = ``odds ratio'' = increased odds of contracting HPV

\end{frame}


%%%%%%
\begin{frame}{Not significant, but important?}

    Recall that ``not signficant'' means
    \begin{quote}
        there was not sufficient evidence that the observed difference was due to anything other than chance variation.
    \end{quote}


    \vspace{2em}

    This does \alert{not} mean that there is no real difference!

    \vspace{2em}

    If the confidence interval includes zero, but also some big (important) values,
    then we don't have the \alert{power} to tell.  
    Here ``not significant'' reflects our uncertainty.



\end{frame}


%%%%%%
\begin{frame}{Tomato yield}

    Comparing the yield of two tomato varieties,
    a mean difference of 1 pound per plant is ``important''.

    \vspace{2em}

    \begin{center}
    \begin{tabular}{c|cc}
        95\% CI & significant? & important? \\
        \hline
        0.2 -- 0.3 & & \\
        1.2 -- 1.3 & & \\
        0.2 -- 1.3 & & \\
        -0.2 -- 0.3 & & \\
        -1.2 -- 0.3 & & \\
        \hline
    \end{tabular}
    \end{center}


\end{frame}


%%%%%%
\mode<presentation>{
\begin{frame}{Tomato yield}

    Comparing the yield of two tomato varieties,
    a mean difference of 1 pound per plant is ``important''.

    \vspace{2em}

    \begin{center}
    \begin{tabular}{c|cc}
        95\% CI & significant? & important? \\
        \hline
        0.2 -- 0.3 & yes & no \\
        1.2 -- 1.3 & yes & yes \\
        0.2 -- 1.3 & yes & don't know \\
        -0.2 -- 0.3 & no & no \\
        -1.2 -- 0.3 & no & don't know \\
        \hline
    \end{tabular}
    \end{center}


\end{frame}
}


\section<article>{Summary}
\section<presentation>*{Summary}

\begin{frame}{Summary}

    When communicating numerical results using statistics:
      \begin{enumerate}
          \item Always think about what you are trying to learn,
          \item what you have learned about, 
          \item and what is still uncertain.
      \end{enumerate}
      (Sometimes you have to do this for others' results as well.)


    \vspace{1em}

    \begin{enumerate}
        \item Statistical signficance \alert{does not imply} real--world significance.
        \item Lack of statistical signficance \alert{does not imply} real--world insignificance.
        \item Sample size acts like a magnifying glass, making smaller differences visible.
        \item The \alert{effect size} is the difference in means as a proportion of the within-population variability, and is
          a good way to describe the scale of the difference.
        \item Confidence intervals are a good way to assess both statistical signficance and real--world importance.
    \end{enumerate}

\end{frame}

% homework
\begin{frame}{Homework}
  \begin{center}

  \vspace{2em}

  7.6.1

  \vspace{2em}

  % 7.5.1

  \vspace{2em}

  \end{center}
\end{frame}






\end{document}





