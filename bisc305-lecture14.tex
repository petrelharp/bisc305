% Copyright 2007 by Till Tantau
%
% This file may be distributed and/or modified
%
% 1. under the LaTeX Project Public License and/or
% 2. under the GNU Public License.
%
% See the file doc/licenses/LICENSE for more details.


\lecture[1]{Syntax versus Semantik}{lecture-text}

\subtitle{Text und seine Bedeutung}

\date{22 October 2013}


\begin{document}

\begin{frame}
  \maketitle
\end{frame}


\section*{Ziele und Inhalt}

\begin{frame}{Die Lernziele der heutigen Vorlesung und der �bungen.} 
  \begin{enumerate}
  \item Die Begriffe Syntax und Semantik erkl�ren k�nnen
  \item Syntaktische und semantische Elemente nat�rlicher Sprachen und
    von Programmiersprachen benennen k�nnen
  \item Die Begriffe Alphabet und Wort kennen
  \item Objekte als Worte kodieren k�nnen
  \end{enumerate}
\end{frame}

\begin{frame}\frametitle<presentation>{Gliederung}
  \tableofcontents
\end{frame}


%%%%%% %%%%%%
\section{Statistical, and real--world, ``signficance''}

%%%%%%
\begin{frame}{Statistical signficance}

    \begin{blockquote}
        the difference was \only<2->{\alert{statistically}} significant \only<2->{\alert{(with $P<.05$)}}
    \end{blockquote}

    \uncover<3->
    means
    \begin{blockquote}
        we found good evidence that the difference in sample means was not caused by chance variation alone
    \end{blockquote}

    \uncover<4->
    or more precisely
    \begin{blockquote}
        if the null hypothesis was true, we'd expect to see at least this large a difference no more than 5\% of the time
    \end{blockquote}

\end{frame}

%%%%%%
\begin{frame}{Statistical insignficance}

    \begin{blockquote}
        the difference was \alert{not} significant (with $P>0.5$)
    \end{blockquote}

    \uncover<3->
    means
    \begin{blockquote}
        there was not sufficient evidence that the observed difference was due to anything other than chance variation
    \end{blockquote}

    \uncover<4->
    or more precisely
    \begin{blockquote}
        if the null hypothesis was true, we'd expect to see at least this large a difference \alert{at least} 5\% of the time
    \end{blockquote}

\end{frame}


%%%%%%
\begin{frame}{Interpret:}

\end{frame}



%%%%%% %%%%%% %%%%%% %%%%%%
\section{Measuring real--world significance}


%%%%%%
\begin{frame}{It's significant, but is it important?}


    \begin{block}{the difference in means}
        We've observed a difference between two populations:
        \[  \bar y_1 - \bar y_2 \]
    \end{block}

    \ldots and got a small $P$-value.  But, do we care?

    \vspace{2em}

    \structure{Ask yourself:} What do we want to know?

\end{frame}


%%%%%% %%%%%%
\subsection{The effect size}

%%%%%%
\begin{frame}{Effect size}

    \begin{block}<2->{the $t$-statistic}
        \begin{align}
            t_s = \frac{\bar y_1 - \bar y_2}{ \SE_{\bar Y_1 - \bar Y_2} } 
            = \frac{ \text{(actual difference)} }{ \text{(magnitude of sampling noise)} }
        \end{align}
        is big if the observed difference is large compared to the estimated size of the \alert{estimation error}.
    \end{block}

    \begin{block}<3->{the effect size}
        If both populations have the same SD, $\sigma$,
        \begin{align}
            \text{(Effect size)} = \frac{\bar y_1 - \bar y_2}{ \sigma }
            = \frac{ \text{(actual difference)} }{ \text{(magnitude of within-population variation)} }
        \end{align}
        is big if the observed difference is large compared to \alert{within-population variation}.
    \end{block}

\end{frame}


%%%%%% %%%%%%
\subsection{Using confidence intervals}

%%%%%%
\begin{frame}{Significantly unimportant?}

    The \alert{confidence interval} describes where we are pretty confident the true value lies.

    \vspace{2em}

    So if we think an effect smaller than $x$ is \alert{unimportant}
    and $x$ is outside the confidence interval,

    \vspace{2em}

    then we have good evidence the true effect is unimportant.
    (inconsequential, insignificant, \ldots?)


    \vspace{2em}

    Example:


\end{frame}


%%%%%%
\begin{frame}{Not significant, but important?}

    Recall that ``not signficant'' means
    \begin{blockquote}
        there was not sufficient evidence that the observed difference was due to anything other than chance variation.
    \end{blockquote}


    \vspace{2em}

    This does \alert{not} mean that there is no real difference!

    \vspace{2em}

    If the confidence interval includes zero, but also some big (important) values,
    then we don't have the \alert{power} to tell.  
    Here ``not significant'' reflects our uncertainty.



\end{frame}


%%%%%%
\begin{frame}{Tomato yield}

    Comparing the yield of two tomato varieties,
    a mean difference of 1 pound per plant is ``important''.

    \begin{tabular}{c|cc}
        95\% CI & significant? & important? \\
        \hline
        0.2 -- 0.3 & & \\
        1.2 -- 1.3 & & \\
        0.2 -- 1.3 & & \\
        -0.2 -- 0.3 & & \\
        -1.2 -- 0.3 & & \\
        \hline
    \end{tabular}


\end{frame}


%%%%%%
\begin{frame}<presentation>{Tomato yield}

    Comparing the yield of two tomato varieties,
    a mean difference of 1 pound per plant is ``important''.

    \begin{tabular}{c|cc}
        95\% CI & significant? & important? \\
        \hline
        0.2 -- 0.3 & yes & no \\
        1.2 -- 1.3 & yes & yes \\
        0.2 -- 1.3 & yes & don't know \\
        -0.2 -- 0.3 & no & no \\
        -1.2 -- 0.3 & no & don't know \\
        \hline
    \end{tabular}


\end{frame}






\section<article>{Summary}
\section<presentation>*{Summary}

\begin{frame}{Summary}
    When doing statistics:
      \begin{enumerate}
          \item Always think about what you are trying to learn,
          \item what you have learned about, 
          \item and what is still uncertain.
          \item Always say what you mean.
      \end{enumerate}
      (Sometimes you have to do this for others' results as well.)


    \vspace{2em}

    \begin{enumerate}
        \item Statistical signficance \alert{does not imply} real--world significance.
        \item Lack of statistical signficance \alert{does not imply} real--world insignificance.
        \item \alert{Never} call the result of a statistical test ``insignificant''.
        \item The effect size is a good way to measure the real-world impact of a difference in means.
        \item Confidence intervals are a good way to assess both statistical signficance and real--world importance.
    \end{enumerate}

\end{frame}

% homework
\begin{frame}{Homework}
  \begin{center}

  \vspace{2em}

  7.6.1

  \vspace{2em}

  % 7.5.1

  \vspace{2em}

  \end{center}
\end{frame}






\end{document}


\begin{frame}{Die zwei Hauptbegriffe der heutigen Vorlesung.}
  \begin{block}{Grobe Definition (Syntax)}
    Unter einer \alert{Syntax} verstehen wir \alert{Regeln}, nach denen
    Texte \alert{strukturiert} werden d�rfen. 
  \end{block}
  \begin{block}{Grobe Definition (Semantik)}
    Unter einer \alert{Semantik} verstehen wir die Zuordnung von
    \alert{Bedeutung} zu Text.
  \end{block}
\end{frame}


\subsection[Syntax \protect\\ nat�rlicher Sprachen]{Syntax nat�rlicher Sprachen}

\begin{frame}{Beobachtungen zu einem �gyptischen Text.}
  \includegraphicscopyright[width=6cm]{beamerexample-lecture-pic3.jpg}
  {Copyright by Guillaume Blanchard, GNU Free Documentation License, Low Resultion}

  \begin{block}{Beobachtungen}
    \begin{itemize}
    \item Wir haben keine Ahnung, was der Text bedeutet.
    \item Es gibt aber \alert{Regeln}, die offenbar eingehalten wurden,
      wie �Hieroglyphen stehen in Zeilen�.
    \item Solche Regeln sind \alert{syntaktische Regeln} -- man kann sie
      �berpr�fen, ohne den Inhalt zu verstehen.
    \end{itemize}
  \end{block}
\end{frame}


% . . . 

\section<article>{Summary}
\section<presentation>*{Summary}

\begin{frame}{Summary}
  \begin{enumerate}
  \item Ein \alert{Wort} ist eine Folge von Symbolen aus einem
    \alert{Alphabet}. 
  \item Eine \alert{Syntax} besteht aus Regeln, nach denen
    Worte (Texte) gebaut werden d�rfen.
  \item Eine \alert{Semantik} legt fest, was Worte \alert{bedeuten}.
  \item Eine \alert{formale Sprache} ist eine Menge von Worten
    �ber einem Alphabet.
  \end{enumerate}
\end{frame}

% homework
\begin{frame}{Homework}
  \begin{center}

  % 7.4.2

  \vspace{2em}

  % 7.4.8

  \vspace{2em}

  % 7.5.1

  \end{center}
\end{frame}


\end{document}





