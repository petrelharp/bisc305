% Copyright 2007 by Till Tantau
%
% This file may be distributed and/or modified
%
% 1. under the LaTeX Project Public License and/or
% 2. under the GNU Public License.
%
% See the file doc/licenses/LICENSE for more details.


\lecture[20]{Inference for proportions}{lecture-text}

\subtitle{the chi-squared statistic}

\date{12 November 2013}

%% pp 348-362

% FROM LAST TIME: 
%% one-sided CIs for Wilson's estimator
%% choosing your SE based on sample size


\begin{document}

\begin{frame}
  \maketitle
\end{frame}

\begin{frame}\frametitle<presentation>{Outline}
  \tableofcontents
\end{frame}


%%%%% %%%%%%% %%%%%%%%
\section{Hypotheses for proportions}

\begin{frame}{Example: Hardy--Weinberg}

    Red--green colorblindness is caused by recessive mutations in one of the opsin genes on the X chromosome.
    The frequency in males (only one X) indicates that these mutations are at around $q=8.3\%$ in Europeans.
    % This suggests the frequency in females should be $q^2= 0.69\%$.
    We sample 1000 females and compare frequencies of nonmutants, carriers, and color-blindness to the expected proportions $(1-q)^2$, $2q(1-q)$, and $q^2$:
    \begin{center}
        \begin{tabular}{r|rr}
            & observed counts & expected counts \\
            \hline 
            nonmutant & 850 & 840.9 \\ 
            carrier &  146 & 152.2 \\ 
            colorblind & 4 & 6.9  
        \end{tabular}
    \end{center}

\end{frame}


%%%%%% %%%%%%%% %%%%%%%
\section{The $\chi^2$ test statistic}

%%%%%%
\begin{frame}{Calculating chi-squared}

    \begin{block}{The $\chi^2$ statistic}
        \[ \chi^2_s = \sum_{i=1}^k \frac{(o_i - e_i)^2}{e_i} , \]
        where $o_i$ is the observed \alert{number} of observations in category $i$,
        and $e_i$ is the expected \alert{number} of observations in category $i$.
    \end{block}


    \vspace{2em}

    This is \alert{``observed minus expected, squared, divided by expected''}.


\end{frame}


%%%%%%
\begin{frame}{Example}

    Colorblindness in females:
    \begin{center}
        \begin{tabular}{r|rrr}
            & observed counts & expected counts & difference\\
            \hline 
            nonmutant & 850 & 840.9 & 9.1 \\ 
            carrier &  146 & 152.2 & -5.8 \\ 
            colorblind & 4 & 6.9 & 2.9  \\
        \end{tabular}
    \end{center}

    \vspace{2em}

    \[
        \chi^2_s = \frac{9.1^2}{840.9} + \frac{5.8^2}{152.2} + \frac{2.9^2}{6.9} .
    \]

\end{frame}


%%%%%%
\begin{frame}{Intuition for the formula}

    In the formula
        \[ \chi^2_s = \sum_{i=1}^k \frac{(o_i - e_i)^2}{e_i} , \]
    why do we divide by the expected number?

    \vspace{2em} \pause

    \begin{enumerate}
        \item Seeing 170 instead of 150 out of 500 is less suprising \\
            than seeing 21 instead of 1 out of 500.
        \item Why not divide by $e_i^2$ instead? \\
            Recall that the standard deviation of the observed number of $p$-successes is $\sqrt{n p(1-p) }$.
    \end{enumerate}

\end{frame}


%%%%%% %%%%%% %%%%%%
\section{The null distribution}

\begin{frame}{Simulate the null distribution}


    \vspace{2em}

    demonstration

    \vspace{2em}

    compare table to simulation


\end{frame}


%%%%%% %%%%%%
\section{What you can conclude}



%%%%%%
\begin{frame}{Another example}

    By area, 
    40\% is pavement,
    30\% of a Santa Monica parking lot is shiny cars,
    20\% is dull cars,
    and 10\% is other.\\
    \structure{Hypothesis:} birds don't notice what they are pooping on.
    We counted the numbers of bird poops on each in a given day:
    \begin{center}
        \begin{tabular}{rr}
            & observed counts \\
            \hline 
            pavement & 65 \\
            shiny & 77 \\
            dull & 35 \\
            other & 18 \\
            \hline
            total & 195
        \end{tabular}
    \end{center}

\end{frame}


%%%%%%
\begin{frame}{What do we conclude?}

    The $\chi^2$ test is ``omnidirectional'' --
    a small $P$-value does not tell us \alert{which part} of the table
    is deviating significantly from the expectation.

    \vspace{2em}

    \structure{Exception:} a \alert{dichotomous} table does allow for directional tests.
    (there's only one proportion)


\end{frame}


%%%%%%
\begin{frame}{Back to the example}
    \begin{center}
        \begin{tabular}{rrrr}
            & observed counts & expected counts & difference \\
            \hline 
            pavement & 65 & 78 & -13 \\
            shiny & 77 & 58.5 & 18.5 \\
            dull & 35 & 39 & -4 \\
            other & 18 & 19.5 & -1.5 \\
            \hline
            total & 195 & 195  & 0 \\
        \end{tabular}
    \end{center}

    \vspace{2em}

    Observations?
\end{frame}


%%%%%%
\begin{frame}{Back to the example}
    Was pavement pooped on less?
    \begin{center}
        \begin{tabular}{rrrr}
            & observed counts & expected counts & difference \\
            \hline 
            pavement & 65 & 78 & -13 \\
            not pavement & 130 & 117 & 13 \\
            \hline
            total & 195 & 195  & 0 
        \end{tabular}
    \end{center}

    \vspace{2em}
    
    \[ \chi^2_s = \frac{ 13^2 }{ 78 } + \frac{ 13^2 }{ 117 } = 3.611 \]
    and $df=1$ so $.02<P<.05$.

    \vspace{2em}

    \structure{Conclusion:} probably yes.


\end{frame}

%%%%%%
\begin{frame}{Back to the example}
    Were shiny cars pooped on more?
    \begin{center}
        \begin{tabular}{rrrr}
            & observed counts & expected counts & difference \\
            \hline 
            shiny & 77 & 58.5 & 18.5 \\
            not shiny cars & 118 & 136.5 & -18.5 \\
            \hline
            total & 195 & 195  & 0 
        \end{tabular}
    \end{center}

    \vspace{2em}
    \[ \chi^2_s = \frac{ 18.5^2 }{ 58.5 } + \frac{ 18.5^2 }{ 136.5 } = 8.36 \]
    and $df=1$ so $.001<P<.01$.

    \vspace{2em}

    \structure{Conclusion:} probably yes.

\end{frame}



\section<article>{Summary}
\section<presentation>*{Summary}

\begin{frame}{Summary}
  \begin{enumerate}
      \item We now have a good test statistic ($\chi^2_s$) to compare observed proportions
          to expected proportions in a sample.
      \item This test statistic has good theoretical properties.
      \item $\chi^2_s$ is equal to the sum of deviations of observed counts from expected counts, squared, and divided by the expected counts.
      \item The null distribution has one parameter, degrees of freedom, equal to the number of counts minus one.
  \end{enumerate}
\end{frame}

% homework
\begin{frame}{Homework}
  \begin{center}

  9.4.1

  \vspace{2em}

  9.4.6


  \end{center}
\end{frame}


\end{document}





