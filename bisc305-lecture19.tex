% Copyright 2007 by Till Tantau
%
% This file may be distributed and/or modified
%
% 1. under the LaTeX Project Public License and/or
% 2. under the GNU Public License.
%
% See the file doc/licenses/LICENSE for more details.


\lecture[19]{Estimating proportions}{lecture-text}

\subtitle{with confidence}

\date{7 November 2013}


\begin{document}

\begin{frame}
  \maketitle
\end{frame}


\begin{frame}{So far we have}
  \begin{enumerate}
    \item analyzed \alert{quantitative} data:
    \item when each observation is a numerical measurement,
    \item using means, standard deviations, ranks, etcetera.
    \item \alert{Categorical data} assigns each observation to a discrete category,
    \item and we analyze the proportions, or counts in each.
  \end{enumerate}
\end{frame}

\begin{frame}\frametitle<presentation>{Outline}
  \tableofcontents
\end{frame}



\section{Estimating the sample proportion}
\subsection{Intuition}

%%%%%%
\begin{frame}{What's your guess?}

    In a random sample of 20 pigeons in Los Angeles,
    we see 5 with brown heads and 15 with grey heads.\\
    \structure{What do you think the true proportion of Los Angeles pigeons with brown heads is?}

    \vspace{2em}

    \pause

    We sample another 80 pigeons, and find \emph{none} with red-polka-dotted heads.\\
    \structure{What do you think the true proportion of Los Angeles pigeons with red-polka-dotted heads is?}

    \vspace{2em}

    \pause

    Some pigeons have white heads due to interbreeding with escaped domestic birds,
    but \emph{none} the 100 pigeons we have so far have white heads.\\
    \structure{What do you think the true proportion of Los Angeles pigeons with white heads is?}


    \structure{Is it reasonable} that the true proportion of pigeons with white heads is 10\%?


\end{frame}

%%%%%%
\begin{frame}{Estimates, and uncertainty}

    We have a random sample of observations falling in two categories.

    \vspace{2em}

    We want to
    \begin{enumerate}
        \item Estimate the population frequency.
        \item Evaluate our uncertainty about that estimate.
    \end{enumerate}

    \vspace{2em}

    If you observe \alert{none} of one category, \\
        should you estimate that the population really has none? \\
            \hspace{3em} \structure{Usually not.}  {\small (else, why were you looking?)}

\end{frame}

\subsection{Estimating the population frequency}

%%%%%%
\begin{frame}{The sampling distribution}

    \begin{block}{Central Limit Theorem for Frequencies}
        If $Y$ is the number of observations of type ``A''
        out of a sample of $n$ independent draws
        from a large population with a true frequency $p$ of type ``A'', \\
        then $Y$ is \alert{approximately Normal},
        with mean $np$ and standard deviation $\sqrt{n p (1-p)}$,
        if $n$ is large enough.
    \end{block}

    \vspace{2em}

    This implies that 
    \[ \prob\left( np - 1.96 \sqrt{n p(1-p)} \le Y \le np + 1.96 \sqrt{n p (1-p)} \right) \approx 0.95 , \]
    i.e.\ the number of observed type ``A'' should be about $np$, give or take $2 \sqrt{n p (1-p)}$ or so.

    \vspace{2em}

    \structure{simulation example}

\end{frame}

%%%%%%
\begin{frame}{The standard error for $\hat p$}

    The \alert{sample proportion} of type A is
    the fraction of the sample that is type A:
    \[ \hat p = \frac{y}{n} . \]

    \vspace{2em}

    By the Central Limit Theorem, we know that 
    \[  \hat p \approx p \pm 2 \sqrt{ \frac{ p(1-p) }{ n } } .\]

    \vspace{2em}

    But, we don't actually know the population proportion $p$.

\end{frame}

%%%%%%
\begin{frame}{An estimator for $p$}

    \begin{block}{Wilson's estimator}
        for the population proportion is
        \[ \wt p = \frac{ y+2 }{ n+4 } ,\]
        i.e.\ the frequency in the sample with two imaginary observations of each type added.
    \end{block}

    \vspace{2em}


    \begin{block}{Standard Error}
        for Wilson's estimator is
        \[ \SE_{\wt P} = \sqrt{\frac{\wt p (1-\wt p)}{n+4} } .\]
    \end{block}


\end{frame}

\subsection{Confidence intervals}

%%%%%%
\begin{frame}{Confidence intervals}

    We know that 
    \begin{itemize}
        \item the sampling distribution of Wilson's estimator $\wt P$ is approximately Normal
        \item with standard error $\SE_{\wt P}$,
        \item so a $(1-\alpha)$ confidence interval is $z_{\alpha/2} \SE_{\wt P}$.
    \end{itemize}
    It turns out that this usually does a good job \\
    \alert{even if $n$ is not large.}

    \vspace{2em}

    \begin{block}{A 95\% confidence interval}
        for Wilson's estimator is
        \[ \wt p \pm 1.96 \; \SE_{\wt P} . \]
    \end{block}

    \vspace{2em}

    Check this, \structure{by simulation}.

\end{frame}


%%%%%%
\begin{frame}{Example}

    Out of 169 women with family histories of breast cancer, 27 had mutations in the BRCA1 gene.

    \vspace{2em}

    \alert{Wilson's estimator} gives that around
        \[ \wt p = \frac{27 + 2}{169+4} = .168 \]
    of women with family histories of breast cancer have mutations in BRCA1.

    \vspace{2em}

    The \alert{standard error} of this estimate is
    \[ \SE_{\wt P} = \sqrt{\frac{.168(1-.168)}{169+4}} = .028  \]

    \vspace{2em}

    A \alert{95\% confidence interval} for this proportion is $\wt p \pm 1.96 \times .028$:
    \[ 0.113 < p < 0.223 \]

\end{frame}

%%%%%%
\begin{frame}{From the wild}

    \begin{center}
    \includegraphicscopyright[width=\textwidth]{barn-owl-comparison}{Owl predation in Pennsylvania, Pearson \& Pearson 1947}
    \end{center}

    \vspace{2em}

    Find confidence intervals for some of these percentages.  
    Is there strong evidence that owls catch a higher percentage than traps of 
    \textit{(a)} meadow mice?
    \textit{(b)} Deer mice?
    \textit{(c)} Star-nosed moles?


\end{frame}

%%%%%%
\begin{frame}{Example}

    Eleven newborn babies with respiratory failure were given ECMO.  None died.

    \vspace{2em}

    \alert{Wilson's estimator} of the probability of death is
    \[ \wt p = \frac{0+2}{11+4} = 0.133 \]

    \vspace{2em}

    The \alert{standard error} of this estimate is
    \[ \SE_{\wt P} = \sqrt{\frac{ .133(1-.133) }{ 11+4 }} = .088 \]

    \vspace{2em}

    A \alert{95\% confidence interval} for this proportion is
        \[ -0.039 \only<2->{ < 0 }< p < 0.305 \]


\end{frame}

%%%%%%
\begin{frame}{One-sided confidence}

    We might often want to simply \alert{upper bound} the population frequency,\\\
    i.e.\ construct a \structure{one-sided confidence interval}.

    \vspace{2em}

    Since $z_{.05} = 1.645$, 
    a \alert{95\% one-sided CI} is 
        \[ p < \wt p + 1.645 \; \SE_{\wt P} \]

    \vspace{2em}

    In the previous example, this is
        \[ p < 0.278 , \]
    so we are 95\% confident that the true chance of death is less than 27.8\%.


\end{frame}

\subsection{Planning studies}

%%%%%%
\begin{frame}{Pick your SE}

    We want to estimate the population percentage to \alert{1\% accuracy}
    (at 95\% confidence).
    How big a sample should we take?

    \vspace{2em}
    \pause

    \structure{Rephrased:} We want to estimate the population percentage
    with two {\tiny (really, 1.96)} standard errors no larger than 1\%.
    How big a sample should we take?

    \vspace{2em}

    Well,
        \[ 2 \SE_{\wt P} = 2 \sqrt{\frac{\wt p (1-\wt p)}{n+4} } ,\]
        so it \alert<1>{depends on $p$}, \pause but \alert<2>{no matter what $p$ is,}\\
    $\sqrt{p (1-p)} \le 1/2$,
    so
    \[ 2 \SE_{\wt P} \le \sqrt{\frac{1}{(n+4)} } .\]

    \vspace{1em}

    We can guarantee that $2 \SE_{\wt P} \le .01$ if $n \ge 100^2$, \\
    i.e.\ we have at least 10,000 observations.


\end{frame}

%%%%%%
\begin{frame}{Example: polling}

    We want to do a poll to determine whether a state proposition 
    is ahead (above 50\%) or not,
    with 99\% confidence,
    as long it is actually leading (or, behind) by at least 5\%.
    How many people should we poll?
    Assume everyone is decided.

\end{frame}


% . . . 

\section<article>{Summary}
\section<presentation>*{Summary}

\begin{frame}{Summary}
  \begin{enumerate}
      \item The best estimator of the population frequency is not necessarily the sample frequency
      \item \ldots mostly because it gets tripped up for small frequencies.
      \item A good choice is \alert{Wilson's estimator} $\wt p$, that adds two imaginary observations of each type.
      \item The standard error of Wilson's estimator is $\sqrt{ \wt p (1-\wt p) / (n+4) }$,
      \item \ldots and confidence intervals for $\wt p$ are found using $z$ multipliers.
      \item Also: always think about what the results mean.
  \end{enumerate}
\end{frame}

% homework
\begin{frame}{Homework}
  \begin{center}

  9.2.2

  \vspace{2em}

  9.2.9

  \vspace{2em}

  9.2.10

  \end{center}
\end{frame}


\end{document}





