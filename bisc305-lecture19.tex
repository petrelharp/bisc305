% Copyright 2007 by Till Tantau
%
% This file may be distributed and/or modified
%
% 1. under the LaTeX Project Public License and/or
% 2. under the GNU Public License.
%
% See the file doc/licenses/LICENSE for more details.


\lecture[19]{The population proportion}{lecture-text}

\subtitle{categorigcal data}

\date{7 November 2013}


\begin{document}

\begin{frame}
  \maketitle
\end{frame}


\begin{frame}{So far we have}
  \begin{enumerate}
    \item analyzed \alert{quantitative} data:
    \item when each observation is a numerical measurement,
    \item using means, standard deviations, ranks, etcetera.
    \item \alert{Categorical data} assigns each observation to a discrete category,
    \item and we analyze the proportions, or counts in each.
  \end{enumerate}
\end{frame}

\begin{frame}\frametitle<presentation>{Outline}
  \tableofcontents
\end{frame}



\section{Estimating the sample proportion}
\subsection{Intuition}

%%%%%%
\begin{frame}{What's your guess?}



\end{frame}


% the sample proportion
%% intuition: what is the true proportion?
%% intuition: how big is it likely to be?
%% answer by simulation
% estimating the sample proportion
%% the sampling distribution
%% recall the CLT
%% SE with known p
%% Wilson's estimator
%% SE for Wilson's estimator
% confidence intervals for the proportion
%% one sided
%% two sided

%%%%%%
\begin{frame}{}

\end{frame}

\begin{frame}{Die zwei Hauptbegriffe der heutigen Vorlesung.}
  \begin{block}{Grobe Definition (Syntax)}
    Unter einer \alert{Syntax} verstehen wir \alert{Regeln}, nach denen
    Texte \alert{strukturiert} werden d�rfen. 
  \end{block}
  \begin{block}{Grobe Definition (Semantik)}
    Unter einer \alert{Semantik} verstehen wir die Zuordnung von
    \alert{Bedeutung} zu Text.
  \end{block}
\end{frame}


\subsection[Syntax \protect\\ nat�rlicher Sprachen]{Syntax nat�rlicher Sprachen}

\begin{frame}{Beobachtungen zu einem �gyptischen Text.}
  \includegraphicscopyright[width=6cm]{beamerexample-lecture-pic3.jpg}
  {Copyright by Guillaume Blanchard, GNU Free Documentation License, Low Resultion}

  \begin{block}{Beobachtungen}
    \begin{itemize}
    \item Wir haben keine Ahnung, was der Text bedeutet.
    \item Es gibt aber \alert{Regeln}, die offenbar eingehalten wurden,
      wie �Hieroglyphen stehen in Zeilen�.
    \item Solche Regeln sind \alert{syntaktische Regeln} -- man kann sie
      �berpr�fen, ohne den Inhalt zu verstehen.
    \end{itemize}
  \end{block}
\end{frame}


% . . . 

\section<article>{Summary}
\section<presentation>*{Summary}

\begin{frame}{Summary}
  \begin{enumerate}
  \item Ein \alert{Wort} ist eine Folge von Symbolen aus einem
    \alert{Alphabet}. 
  \item Eine \alert{Syntax} besteht aus Regeln, nach denen
    Worte (Texte) gebaut werden d�rfen.
  \item Eine \alert{Semantik} legt fest, was Worte \alert{bedeuten}.
  \item Eine \alert{formale Sprache} ist eine Menge von Worten
    �ber einem Alphabet.
  \end{enumerate}
\end{frame}

% homework
\begin{frame}{Homework}
  \begin{center}

  % 7.4.2

  \vspace{2em}

  % 7.4.8

  \vspace{2em}

  % 7.5.1

  \end{center}
\end{frame}


\end{document}





