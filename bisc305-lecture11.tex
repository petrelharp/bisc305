% Copyright 2007 by Till Tantau
%
% This file may be distributed and/or modified
%
% 1. under the LaTeX Project Public License and/or
% 2. under the GNU Public License.
%
% See the file doc/licenses/LICENSE for more details.


\lecture[11]{Comparing two samples}{lecture-text}

\subtitle{hypothesis testing, and the $t$ statistic}

\date{2 March 2015}

% pp. 218-(242?)


\begin{document}

\begin{frame}
  \maketitle
\end{frame}


\begin{frame}\frametitle<presentation>{Outline}
  \tableofcontents
\end{frame}


\section{Hypothesis testing}

%%%%%%
\begin{frame}{The goal}

    You have measurements from two groups of samples:
    \begin{itemize}
        \item case/control
        \item treatment/control
        \item red/green
        \item (etcetera)
    \end{itemize}

    \vspace{2em}

    They have different means:
    \[  \bar y_1 > \bar y_2 . \]
    Do we believe the groups are different?

    \vspace{2em}
    \pause

    \begin{block}{If the label has no effect,}
        we can the samples are from the \alert{same population}.
    \end{block}

    \vspace{2em}
    \pause

    \structure{Randomizing} the \alert{group labels}
    estimates how big a difference $|\bar y_1 - \bar y_2|$ is expected 
    due to random chance.

\end{frame}


%%%%%
\begin{frame}{Example: }

    Flexibility (in cm):  
    \begin{center}
      \begin{tabular}{rcrr}
        & & Aerobics & Dance \\
       \hline
       & 38  & 48 \\
       & 45  & 59 \\
       & 58  & 61 \\
       & 64  & \\
       \hline
       mean & 51.25 & 56.00 \\
     \end{tabular}
   \end{center}
     A 10\% difference!  Do we believe it?

     \vspace{2em}

     \pause

     \structure{Proportion of} rearrangements with
     \[
         | \bar y_1 - \bar y_2 | > |51.25-56.00| = 4.75
     \]
     is $20/35 = 0.57$.
 
\end{frame}


%%%%%
\begin{frame}{Example:}

    Are earthquakes bigger around solistices and equinoxes?

\end{frame}


%%%%%
\begin{frame}{Hypothesis testing}

    Here's what we just did:
    \begin{itemize}
        \item No difference in means between groups\\
            $\Rightarrow$ expect \structure{(this much)} difference in sample means\\
            due to random chance.
        \item If the observed difference \alert{seems unlikely} \\
            there's probably a real difference.
    \end{itemize}

     \vspace{2em}
     \pause

     Or:
     \begin{itemize}
         \item Either
             \begin{itemize}
                 \item[(null)] The observed difference is just due to chance, or
                 \item[(alternative)] the population means of the two groups are different.
             \end{itemize}
         \item If the observed difference has \alert{small probability}
             in the case that $H_0$ is true, that's good evidence for $H_A$.
     \end{itemize}

     \vspace{2em}
     \pause

     Or:
     \begin{itemize}
         \item Either
             \begin{itemize}
                 \item[$H_0$] $\mu_1 = \mu_2$, or
                 \item[$H_A$] $\mu_1 \neq \mu_2$.
             \end{itemize}
         \item Assuming $H_0$, if
             \[ \text{Prob}( |\bar Y_1 - \bar Y_2| \ge |\bar y_1 - \bar y_2| ) \]
             is small, then we \alert{reject} $H_0$.
     \end{itemize}

\end{frame}


%%%%%
\begin{frame}{Framing hypotheses}

    We had:
     \begin{itemize}
         \item[$H_0$] The observed difference is just due to chance.
         \item[$H_A$] The population means of the two groups are different.
     \end{itemize}

     \vspace{2em}

    The general framework is:
         \begin{itemize}
             \item[$H_0$] I'm worried that my results might just be caused by random noise of this sort.
             \item[$H_A$] But if not, I have good evidence that this is true.
         \end{itemize}

\end{frame}

%%%%%% %%%%%%
\section{The $t$ test}


%%%%%
\begin{frame}{}

    To test the null hypotheses
    \[  H_0: \qquad \only<1>{\mu_1 = \mu_2} \only<2->{\mu_1 - \mu_2 = 0} \]
    against the alternative
    \[  H_A: \qquad \only<1>{\mu_1 \neq \mu_2} \only<2->{\mu_1 - \mu_2 \neq 0} \]
     \pause
     \pause
    we can use

     \vspace{2em}

     \begin{block}{The $t$ test (short form)}
         Under $H_0$, the \alert{$t$ statistic}
         \[ t_s = \frac{ (\bar y_1 - \bar y_2) - 0 }{ \SE_{\bar Y_1 - \bar Y_2} } \]
         has, approximately, \alert{Student's $t$ distribution}
         with degrees of freedom equal to
         \[ \df = \frac{ (\SE_1^2 + \SE_2^2)^2 }{ \SE_1^4/(n_1-1) + \SE_2^4/(n_2-1) } . \]
     \end{block}

\end{frame}

%%%%%
\begin{frame}{Example: }

    Noreprenephrin (NE) concentration (ng/gm) in rat brains:
    \begin{center}
      \begin{tabular}{c|rr}
            & Toluene & Control \\
          \hline
          $n$ & 6 & 5 \\
          $\bar y$ & 541.8 & 444.2 \\
          $s$  & 66.1 & 69.6 \\
          $\SE$ & 27 & 31 \\
     \end{tabular}
   \end{center}


     \pause
     \begin{align*}
         \SE_{\bar Y_1 - \bar Y_2} &= 41.195 \\
         t_s &= 2.34  \\
         \df &= 8.47 
     \end{align*}
     $P$-value: 0.0454 .

\end{frame}


%%%%%
\begin{frame}{Example:}


    Are earthquakes bigger around solistices and equinoxes?

    \pause

    Which is ``better'', randomization or the $t$-test?

\end{frame}

%%%%%
\begin{frame}{Assumptions for the $t$ test:}

     \begin{block}{The $t$ test (longer form)}
         If $\bar y_1$ and $\bar y_2$ are independent samples of size $n_1$ and $n_2$
         from normally distributed populations,
         then the \alert{$t$ statistic}
         \[ t_s = \frac{ (\bar y_1 - \bar y_2) - 0 }{ \SE_{\bar Y_1 - \bar Y_2} } \]
         has, Student's $t$ distribution
         with degrees of freedom approximately equal to
         \[ \df = \frac{ (\SE_1^2 + \SE_2^2)^2 }{ \SE_1^4/(n_1-1) + \SE_2^4/(n_2-1) } . \]
     \end{block}

     \vspace{2em}

     \structure{To check,} look at histograms.  (we'll come back to this later)
 
\end{frame}


%%%%%
\section{$P$-values}



%%%%%
\begin{frame}{What is a $P$-value?}

    \begin{center}
        \includegraphics{width=0.6\textwidth}{p_values}
        \figcaption{xkcd:1478}
    \end{center}

\end{frame}



\section<article>{Summary}
\section<presentation>*{Summary}

\begin{frame}{Summary}
  \begin{enumerate}
      \item 
  \end{enumerate}
\end{frame}

% homework
\begin{frame}{Homework}
  \begin{center}

      Find an interesting study reported on in the news.

    \vspace{2em}


  \end{center}
\end{frame}


\end{document}





