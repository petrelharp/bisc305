% Copyright 2007 by Till Tantau
%
% This file may be distributed and/or modified
%
% 1. under the LaTeX Project Public License and/or
% 2. under the GNU Public License.
%
% See the file doc/licenses/LICENSE for more details.


\lecture[review]{Sum-up and revew}{lecture-body}

\date{5 December 2013}


\begin{document}

\begin{frame}\frametitle<presentation>{Outline}
  \begin{enumerate}
    \item General principles
    \item Test
    \item What you should know for each test
    \item Other concepts
  \end{enumerate}
\end{frame}


%%%%%%
\begin{frame}{General Principles}

  \begin{itemize}
    \item always look at, and think about, the data
    \item correlation is not causation (consider confounding factors)
    \item  sample size is like a magnifying glass, it lets you see smaller effect sizes
    \item  ... but statistical significance does not imply real-world importance
    \item  statistical (hypothesis) tests have two pieces: a test statistic, and a null model
    \item  the P-value is the probability of seeing a test statistic at least as extreme under the null model
    \item  nonindependent observations can drastically mislead you
    \item  if the sampling distribution is not close to Normal, use a distribution-free test
  \end{itemize}


\end{frame}


%%%%%%
\begin{frame}{Tests}

  \begin{itemize}
    \item two-sample t-test
    \item Wixocon-Mann-Whitney
    \item paired-sample t
    \item paired-sample sign
    \item Wilcoxon sign-rank
    \item CI for estimated proportion
    \item chi-squared for single-sample proportions
    \item two-sample chi-squared
    \item F-test for one-way ANOVA
    \item F-test for two-way ANOVA
    \item t-test for no linear correlation
  \end{itemize}

\end{frame}

%%%%%%
\begin{frame}{What you should know for each test}

  \begin{itemize}
    \item what sort of data the test applies to
    \item what conditions the test needs to be appropriate
    \item how to check the conditions are satisfied
    \item how to formulate the null \& alternative hypotheses
    \item what the test statistic is and how to compute it
    \item how to interpret the result, statistically, and in real-world terms
  \end{itemize}


\end{frame}

%%%%%%
\begin{frame}{Other concepts}
  
  \begin{itemize}
    \item standard deviation of --
    \item standard error of --
    \item effect size (of difference in means)
    \item back-calculation of necessary sample size to have power to see a certain effect (e.g. for differences in proportion)
    \item the two-way ANOVA model: group means, interaction means, plus noise/variability
    \item correlation: proportion of variance explained
    \item the linear (regression) model: linear conditional men plus noise/variability
    \item regression: residuals, leverage, influence
  \end{itemize}


\end{frame}

\end{document}
