% Copyright 2007 by Till Tantau
%
% This file may be distributed and/or modified
%
% 1. under the LaTeX Project Public License and/or
% 2. under the GNU Public License.
%
% See the file doc/licenses/LICENSE for more details.


\lecture[26]{Using linear regression}{lecture-text}

\subtitle{confidence intervals, outliers, and leverage}

\date{5 December 2013}

% pp. 505-527


\begin{document}

\begin{frame}
  \maketitle
\end{frame}


\begin{frame}\frametitle<presentation>{Outline}
  \tableofcontents
\end{frame}


\section{The linear model}

%%%%%%
\begin{frame}{Conditional means, SDs}

  \begin{block}{Notation}
    \begin{align*}
      \mu_{Y|X} &= \; \text{mean value of $Y$ given $X$} \\
      \sigma_{Y|X} &= \; \text{SD of $Y$ given $X$} \\
    \end{align*}
  \end{block}


    \vspace{2em}

    \structure{Example:}
    Drug dose ($X$) and blood pressure ($Y$) in patients. \\
    $\mu_{Y|X}$ is the mean blood pressure among patients with dose $X$; \\
    $\sigma_{Y|X}$ is the SD.

\end{frame}

%%%%%%
\begin{frame}{The linear model}

  \structure{In words,} the values of $Y$ are normally distributed with mean
  a linear function of $X$ and a standard deviation that does not depend on $X$:
  \begin{align*}
    \mu_{Y|X} &= \beta_0 + \beta_1 X \\
    \sigma_{Y|X} &= \sigma_\epsilon
  \end{align*}
  or, equivalently,
  \begin{align*}
    Y = \beta_0 + \beta_1 X + \epsilon \\
    \epsilon \; \text{independent, with SD $\sigma_\epsilon$} .
  \end{align*}

\end{frame}


%%%%%%
\begin{frame}{Why linear?}

  \begin{enumerate}
      
    \item It is arguably the simplest possible relationship between $X$ and $Y$,
      requiring only two parameters ($b_0$ and $b_1$)

    \item Most things look linear at the right level of approximation.

    \item The mathematics works out nicely.

    \item Nonlinear relationships can often be transformed to linear relationships.

  \end{enumerate}

\end{frame}

%%%%%%
\begin{frame}{Transformation example}

\end{frame}



\section{Estimates from data}


%%%%%%
\begin{frame}{Conditions}

  We clearly need some sort of \structure{independence} in the $(X,Y)$ data to make good estimates.

    \vspace{2em}

    \begin{block}{Random Subsampling Model}
      Each observed pair $(x_i,y_i)$ can be regarded as having $y_i$ sampled
      at random from the \alert{conditional} population of $Y$ values having the $X$ value $x_i$,
      and independent of the other samples.
    \end{block}

    \vspace{2em}

    \structure{Examples:}
    \begin{itemize}
      \item Heights ($X$), weights ($Y$) of a random sample from a populations.
      \item Clotting rate ($Y$) after a dose ($X$) of warfarin was administered,
        with dose assigned randomly.
    \end{itemize}

\end{frame}

%%%%%%
\begin{frame}{Parameter estimates}

  If the random subsampling model holds, then
  \begin{enumerate}
    \item[$b_0$] is an estimate of $\beta_0$ (the intercept)
    \item[$b_1$] is an estimate of $\beta_1$ (the slope)
    \item[$s_e$] is an estimate of $\sigma_\epsilon$ (the SD of the noise)
  \end{enumerate}

\end{frame}

\section{Prediction}

%%%%%%
\begin{frame}{Example:}

  Son's height ($Y$) and ``midparent'' height (average of: father's and 1.08 $\times$ mother's heights)
  (Galton 1885):
  \begin{center}
    \includegraphics<1>{galton.pdf}
    \includegraphics<2>{galton-mean.pdf}
    \includegraphics<3>{galton-pred.pdf}

  \only<2>{\structure{Error} size $\approx \sigma_Y$}
  \only<3>{\structure{Error} size $\approx \sigma_\epsilon < \sigma_Y$}
  \begin{align*}
    \mu_X &= 68.09 & \sigma_X &= 2.52 \\
    \mu_Y &= 68.31 & \sigma_Y &= 1.79 \\
    r &= 0.459 & s_e &= 2.24 
  \end{align*}
  \end{center}

\end{frame}


\section{Inference on the slope}

%%%%%%
\begin{frame}{The standard error of the slope}

  Recall that $b_1$ is our \structure{estimate} for the slope $\beta_1$.

  \begin{block}{Standard Error for $b_1$}
  \end{block}

\end{frame}

% inference on the slope
%   SE of b_1
%   depends on X and Y sampling: fig 12.5.1
%   CI for b_1 using t-distribution
%   hypothesis test for b_1 = 0
% interpretation
%   beware of
%     nonlinearity
%     outliers
%     leverage points
%   residual plot
%   fitted vs residuals
%   transformations

% . . . 

\section<article>{Summary}
\section<presentation>*{Summary}

\begin{frame}{Summary}
  \begin{enumerate}
  \item Ein \alert{Wort} ist eine Folge von Symbolen aus einem
    \alert{Alphabet}. 
  \item Eine \alert{Syntax} besteht aus Regeln, nach denen
    Worte (Texte) gebaut werden d�rfen.
  \item Eine \alert{Semantik} legt fest, was Worte \alert{bedeuten}.
  \item Eine \alert{formale Sprache} ist eine Menge von Worten
    �ber einem Alphabet.
  \end{enumerate}
\end{frame}

% homework
\begin{frame}{Homework}
  \begin{center}

  % 7.4.2

  \vspace{2em}

  % 7.4.8

  \vspace{2em}

  % 7.5.1

  \end{center}
\end{frame}


\end{document}





