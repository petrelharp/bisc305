
\lecture[15]{More on hypothesis testing}{lecture-text}

\subtitle{checking conditions}

\date{24 October 2013}


\begin{document}

\begin{frame}
  \maketitle
\end{frame}


\begin{frame}{Last time}
  What is, and isn't, implied by ``statistically significant'' (or not).
  \begin{enumerate}
    \item Statistical significance means an observed effect probably wasn't the result of chance in sampling,
    \item   \ldots and lack of statistical significance means that it could have been.
    \item The $P$-value says nothing about how big the effect actually is,
    \item   since the sample size ``acts like a magnifying glass''.
    \item The \alert{effect size} is one way to think about how important an effect is.
    \item Confidence intervals can communicate both statistical significance,
      and bounds on the absolute size of an effect.
  \end{enumerate}
\end{frame}

\begin{frame}\frametitle<presentation>{Outline}
  \tableofcontents
\end{frame}


\section{Conditions for (best) use of a $t$-test}

\begin{frame}{What are ``conditions''?}

  Statistical tests let you find the probability that 
  \begin{itemize}
    \item something happens,
    \item under a certain generative model.
  \end{itemize}

  \vspace{2em}

  \structure{Example:} The $P$-value for the $t$-test is the probability that 
  \begin{itemize}
    \item the difference in sample means between indpendent samples from two populations is at least as big as the observed value ($\bar y_1 - \bar y_2$), 
    \item if the two populations have the same mean ($\mu_1 = mu_2$), and the sampling distribution of the sample mean is Normal.
  \end{itemize}

  \vspace{2em}
  
  \alert{Conditions}, a.k.a.\ ``assumptions'', come from the second point.  
  If they are not true, we might
  \begin{itemize}
    \item have \alert{wrong} $P$-values (i.e.\ misreport the probability of type I error)
    \item \structure{and/or} have lower power than a better test (i.e.\ misestimate the probability of type II error)
  \end{itemize}

\end{frame}

\subsection{The $t$-test}

%%%%%%
\begin{frame}{Conditions for the $t$-test}

  \begin{block}{Conditions}
    \begin{enumerate}
      \item The sampling procedure provides:
        \begin{enumerate}
          \item random, independent samples from large populations,
          \item with the two samples independent of each other.
        \end{enumerate}
      \item The sampling distributions of $\bar Y_1$ and $\bar Y_w$ are
        \begin{enumerate}
          \item close enough to Normal.
      \end{enumerate}
    \end{enumerate}
  \end{block}

  \vspace{2em}

  Sampling distributions of sample means are close to Normal if the sample sizes are large enough.\\
  \alert{Large enough} is at least 20, but is larger, 
  the further the \alert{population distributions} are from Normal.

\end{frame}

%%%%%%
\begin{frame}{Simple example}

  \structure{Last time} we saw:

  \vspace{2em}


  Body weight:
  \begin{center}
    \begin{tabular}{crr}
       & Males & Females \\
       \hline
       $n$ & 2 & 2 \\
       $\bar y$ & 175 & 143 \\
       $s$ & 35 & 34
     \end{tabular}

   \vspace{2em}

   {$t_s=0.93$ and $P=0.45$}\\
   CI for $\bar \mu_1 - \bar \mu_2$: $(-117,181)$
   \end{center}

   \vspace{2em}
  
   \alert{Are there any red flags here?}
   
\end{frame}


%%%%%%
\begin{frame}{ }

\end{frame}

% . . . 

\section<article>{Summary}
\section<presentation>*{Summary}

\begin{frame}{Summary}
  \begin{enumerate}
  \item Ein \alert{Wort} ist eine Folge von Symbolen aus einem
    \alert{Alphabet}. 
  \item Eine \alert{Syntax} besteht aus Regeln, nach denen
    Worte (Texte) gebaut werden d�rfen.
  \item Eine \alert{Semantik} legt fest, was Worte \alert{bedeuten}.
  \item Eine \alert{formale Sprache} ist eine Menge von Worten
    �ber einem Alphabet.
  \end{enumerate}
\end{frame}

% homework
\begin{frame}{Homework}
  \begin{center}

  % 7.4.2

  \vspace{2em}

  % 7.4.8

  \vspace{2em}

  % 7.5.1

  \end{center}
\end{frame}


\end{document}





