% Copyright 2007 by Till Tantau
%
% This file may be distributed and/or modified
%
% 1. under the LaTeX Project Public License and/or
% 2. under the GNU Public License.
%
% See the file doc/licenses/LICENSE for more details.


\lecture[18]{Paired sample sign test}{lecture-text}

\subtitle{and the Wilcoxon signed-rank test}

\date{5 November 2013}

\begin{document}

\begin{frame}
  \maketitle
\end{frame}


\begin{frame}{We know how to}
  \begin{enumerate}
    \item Do distribution-free single-sample and two-sample tests.
    \item Use the $t$-test to analyze paired-sample data.
  \end{enumerate}
\end{frame}

\begin{frame}\frametitle<presentation>{Outline}
  \tableofcontents
\end{frame}


\section{The sign test}

\subsection{A test without measurements}

\begin{frame}{Example}

  \structure{Hypothesis:} writing makes your fingers longer by stretching out your ligaments.

    \vspace{2em}
  
  Class survey: how many have
  \begin{enumerate}
    \item middle finger longer on your writing hand?
    \item middle finger shorter on your writing hand?
  \end{enumerate}

    \vspace{2em}

    Supplemental survey: flip a fair coin.  How many got:
  \begin{enumerate}
    \item heads?
    \item tails?
  \end{enumerate}

    \vspace{2em}

    \alert{Test statistic} for the sign test: number of people with longer finger on the writing hand.

\end{frame}


%%%%%%
\begin{frame}{A numerical example}

  Do different antibiotics affect the rate of mutation accumulation of bacteria?
  Sequence genome of bacterial strains; culture in paired petri dishes for 1 week with antibiotics A \& B; sequence again; count the mutations.

    \vspace{1em}

    \noindent
    \structure{$H_0$:} rate of accumulation is the same for both.\\
    \structure{$H_A$:} bacteria accumulate more in one than in the other.

    \vspace{1em}


    \only<1>{
        Number of mutations:
  \begin{center}
  \begin{tabular}{rrrl}
    \hline
  dish & A & B & sign \\ 
    \hline
    1 &   1 &   5 & + \\ 
    2 &   0 &   1 & + \\ 
    3 &   1 &   2 & + \\ 
    4 &   0 &   1 & + \\ 
    5 &   2 &   4 & + \\ 
    6 &   0 &   5 & + \\ 
    7 &   5 &   2 & - \\ 
    8 &   2 &   2 & + \\ 
    9 &   1 &   3 & + \\ 
    10 &   1 &   2 & + \\ 
     \hline
  \end{tabular}
  \end{center}
  }

  \only<2>{
      \begin{center}
      \begin{tabular}{rrrr}
        \hline
        & A & B & total\\ 
        \hline
        \# with more mutations  &   1 &   9 & 10 \\ 
         \hline
      \end{tabular}
      \end{center}

    \vspace{2em}

    $N_+$ is number of differences with $D>0$; $N_-$ is number with $D<0$; 
    \alert{test statistic} $B_s$ is larger of $N_+$, $N_-$.

        \vspace{1em}

        \begin{align*}
            P&= \text{probability that in 10 coin tosses, 9 coins agree.} \\
             &= 0.039
        \end{align*}
    }


\end{frame}

%%%%%% %%%%%%%%%
\subsection{The binomial distribution}

%%%%%%
\begin{frame}{$P$-values from the binomial}

  The null hypothesis is \\
  \hspace{2em} \structure{$H_0$:} difference between the observations is equally likely to be positive or negative.

    \vspace{2em}

    so the $P$-value of the last example \\
    (9 out of 10 pairs agree, calculated with $H_0$) \\
    is:
    \begin{align*} 
      P &= \mbox{Prob}( \; \text{at least 9 heads out of 10 coin flips}\; ) \\
       &\quad + \mbox{Prob}( \; \text{at least 9 tails out of 10 coin flips}\; ) \\
       &= \ch{10}{9} \; 0.5^{10} + \ch{10}{10} \; 0.5^{10} \\
       &\quad + \ch{10}{9} \; 0.5^{10} + \ch{10}{10}\;  0.5^{10} \\
       &= 0.001953125 + 0.017578125 + 0.017578125 + 0.001953125  \\
       &= 0.0390625
     \end{align*}

\end{frame}

%%%%%%
\begin{frame}{Reminder: binomial probabilities}

  \begin{align*}
    &\mbox{Prob}( k \; \text{heads in} \; n \; \text{coin flips} ) \\
    \qquad \qquad &= \ch{n}{k} \; 0.5^k \; 0.5^{n-k} \\
    \qquad \qquad &= \text{(\# ways to get $k$ heads)} \text{(prob of each possibility)}
  \end{align*}

    \vspace{2em}

    \structure{live simulation example}


\end{frame}

%%%%%%
\begin{frame}{What to do about zeros?}

  Again, the null hypothesis is \\
  \hspace{2em} \structure{$H_0$:} difference between the observations is equally likely to be positive or negative.

    \vspace{2em}

    \structure{So:} got some ties?  (i.e.\ $D=0$) \\
    \alert{Ignore them;} compute the $P$-value as if they weren't there.

    \vspace{2em}

    \structure{Because:} we test the hypothesis that \alert{if the paired observations are different}, $Y_1>Y_2$ is just as likely as $Y_2>Y_1$.  (and so, chance of $Y_1>Y_2$, given $Y_1 \neq Y_2$, is 1/2).

\end{frame}


%%%%%%
\begin{frame}{Example}

        Number of mutations:
  \begin{center}
  \begin{tabular}{rrrl}
    \hline
  dish & A & B & sign \\ 
    \hline
    1 &   1 &   5 & + \\ 
    2 &   0 &   0 & 0 \\ 
    3 &   1 &   2 & + \\ 
    4 &   0 &   1 & + \\ 
    5 &   2 &   2 & 0 \\ 
    6 &   0 &   5 & + \\ 
    7 &   5 &   2 & - \\ 
    8 &   2 &   2 & + \\ 
    9 &   1 &   3 & + \\ 
    10 &   1 &   2 & + \\ 
     \hline
  \end{tabular}
  \end{center}

\end{frame}


%%%%%% %%%%%%
\subsection{Extending the idea}

%%%%%%
\begin{frame}{Using the binomial formula}

  Yeast ({\it Saccaromyces cerevisiae}) can be either haploid (one copy of each chromosome per cell) or diploid (two copies, like us).
  (\structure{$H_A$}:) Do haploid yeast accumulate mutations more quickly than diploid yeast (per chromosome)?
  We grow pairs of otherwise identical haploid and diploid yeasts for one generation,
  and counted mutations on each chromosome.

    \vspace{2em}

    In 6 of the 10 pairs, the haploid chromosome had more mutations than both of the diploid chromosomes.
    Is this strong support for $H_A$? {\small (\alert{hint:} this is like the sign test, but the coin is not fair.) }
    % 1-pbinom(q=6,size=10,prob=1/3)


\end{frame}

%%%%% %%%%%%% %%%%%%%%
\section{The Wilcoxon signed-rank test}

\subsection{The signed-rank statistic}

%%%%%%
\begin{frame}{Signed ranks}

    The Wilcoxon signed-rank statistic, for paired sample data,
    \begin{itemize}
        \item orders the \alert{absolute values} of the differences
        \item compares the ranks of the pairs with $D>0$ to the ranks of the pairs with $D<0$.
    \end{itemize}

    \vspace{2em}

    $W_+$ is the sum of the ranks of the pairs with $D>0$; $W_-$ is the sum of the ranks with $D<0$; \\
    the \alert{test statistic}, $W_s$ is the larger of the two.

    \vspace{2em}

    Because it only uses \alert{ranks},
    it is distribution-free.

\end{frame}

%%%%%%
\begin{frame}{Example}
    Density of nerves at two sites in the intestine in 9 horses:
    \begin{center}
\begin{tabular}{rrrr}
  \hline
 animal & site I & site II & difference \\ 
  \hline
  1 &  50.60 & 38.00 & +12.60 \\ 
  2 &  39.20 & 18.60 & +20.60 \\ 
  3 &  35.20 & 23.20 & +12.00 \\ 
  4 &  17.00 & 19.00 & -2.00 \\ 
  5 &  11.20 & 6.60 & +4.60 \\ 
  6 &  14.20 & 16.40 & -2.20 \\ 
  7 &  24.20 & 14.40 & +9.80 \\ 
  8 &  37.40 & 37.60 & -0.20 \\ 
  9 &  35.20 & 24.40 & +10.80 \\ 
   \hline
\end{tabular}
    \end{center}

\end{frame}

\subsection{Caveats}


%%%%%%
\begin{frame}{What to do about zeros? ties?}

    The Wilcoxon signed-rank test deals with caveats like both
    the sign test and the Wilcoxon--Mann--Whitney test: \\
    \begin{itemize}
        \item Omit pairs with no difference. ($D=0$)
        \item Average the ranks of ties.
    \end{itemize}
    \structure{note:} like WMW, the $P$-value is approximate.

    \vspace{2em}

    \structure{Example:}
    \begin{center}
\begin{tabular}{ccccc}
  \hline
  animal & site I & site II & difference & rank \\ 
  \hline
  1 &  1 & 1 & 0 & omit \\ 
  2 &  2 & 3 & +1 & 1.5 \\ 
  3 &  2 & 0 & -2 & 3 \\ 
  4 &  3 & 2 & -1 & 1.5 \\ 
  5 &  2 & 5 & +3 & 4 \\ 
   \hline
\end{tabular}
    \end{center}

    \vspace{2em}

    \begin{gather*}
        W_+ = 5.5  \qquad
        W_- = 4.5 \\
        n = 4 \qquad
        W_s = 5.5 
    \end{gather*}

\end{frame}

%%%%%% %%%%%%%%% %%%%%%%%%%
\subsection{Comparison of the tests}

%%%%%%
\begin{frame}{Conditions}

  The sign test versus the Wilcoxon signed-rank test:
  \begin{enumerate}
    \item Both are distribution--free
    \item The sign test is \alert{more flexible} 
    \item The Wilcoxon signed-rank test is \alert{more powerful} \\
        if the distributions are symmetric
  \end{enumerate}

    \vspace{2em}

    Sign test:
    \[ H_0: \quad \mbox{Prob}(D>0) = 1/2 . \]

    \vspace{2em}

    Wilcoxon signed-rank test:
    \[ H_0: \quad \text{ distribution of $D$ is symmetric } .\]

\end{frame}


\section<article>{Summary}
\section<presentation>*{Summary}

\begin{frame}{Summary}
  \begin{enumerate}
      \item We now have two more ways to compare typical measurement values between two samples.
      \item The \alert{sign test} is very flexible, and doesn't even require numeric measurements.
      \item \ldots count how many pairs have $A>B$.
      \item The Wilcoxon signed-rank test requires the distributions to be symmetric, but is more powerful than the sign test if this is true.
      \item \ldots add the ranks of the pairs with $A>B$.
      \item Both are distribution-free.
  \end{enumerate}
\end{frame}

% homework
\begin{frame}{Homework}
  \begin{center}

  8.4.3

  \vspace{2em}

  8.4.11

  \vspace{2em}

  8.5.4


  \end{center}
\end{frame}


\end{document}





