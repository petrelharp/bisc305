% Copyright 2007 by Till Tantau
%
% This file may be distributed and/or modified
%
% 1. under the LaTeX Project Public License and/or
% 2. under the GNU Public License.
%
% See the file doc/licenses/LICENSE for more details.


\lecture[18]{Paired sample sign test}{lecture-text}

\subtitle{and the Wilcoxon signed-rank test}

\date{5 November 2013}

\begin{document}

\begin{frame}
  \maketitle
\end{frame}


\begin{frame}{We know how to}
  \begin{enumerate}
    \item Do distribution-free single-sample and two-sample tests.
    \item Use the $t$-test to analyze paired-sample data.
  \end{enumerate}
\end{frame}

\begin{frame}\frametitle<presentation>{Outline}
  \tableofcontents
\end{frame}


\section{The sign test}

\subsection{A test without measurements}

\begin{frame}{Example}

  \structure{Hypothesis:} writing makes your fingers longer by stretching out your ligaments.

    \vspace{2em}
  
  Class survey: how many have
  \begin{enumerate}
    \item middle finger longer on your writing hand?
    \item middle finger shorter on your writing hand?
  \end{enumerate}

    \vspace{2em}

    Supplemental survey: flip a fair coin.  How many got:
  \begin{enumerate}
    \item heads?
    \item tails?
  \end{enumerate}

\end{frame}

%%%%%%
\begin{frame}{A nondirectional example}

  Do different antibiotics affect the rate of mutation accumulation of bacteria?
  Sequence genome of bacterial strains; culture in paired petri dishes for 1 week with antibiotics A \& B; sequence again; count the mutations.

    \vspace{2em}

    \structure{$H_0$:} rate of accumulation is the same for both.
    \structure{$H_A$:} bacteria accumulate more in one than in the other.


  \begin{center}
  \begin{tabular}{rrrl}
    \hline
  dish & A & B & sign \\ 
    \hline
    1 &   1 &   5 & + \\ 
    2 &   0 &   1 & + \\ 
    3 &   1 &   2 & + \\ 
    4 &   0 &   1 & + \\ 
    5 &   2 &   4 & + \\ 
    6 &   0 &   5 & + \\ 
    7 &   5 &   2 & - \\ 
    8 &   2 &   2 & + \\ 
    9 &   1 &   3 & + \\ 
    10 &   1 &   2 & + \\ 
     \hline
  \end{tabular}
  \end{center}

    \vspace{2em}

    $P$-value $=$ probability that in 10 coin tosses, get $\ge 9$ or $\le 1$ heads.\\
    $= 0.039$.


\end{frame}

%%%%%% %%%%%%%%%
\subsection{The binomial distribution}

%%%%%%
\begin{frame}{$P$-values from the binomial}

  The null hypothesis is \\
  \hspace{2em} $H_0$: difference between the observations is equally likely to be positive or negative.

    \vspace{2em}

    so the $P$-value of the last example (9 ``+'' out of 10 pairs, calculated with $H_0$) is
    \begin{align*} 
      P &= \mbox{Prob}( \; \text{at least 9 heads out of 10 coin flips}\; ) \\
       &\quad + \mbox{Prob}( \; \text{at least 9 tails out of 10 coin flips}\; ) \\
       &= \ch{10}{9} \; 0.5^10 + \ch{10}{10} \; 0.5^{10} \\
       &\quad + \ch{10}{9} \; 0.5^{10} + \ch{10}{10}\;  0.5^{10} \\
       &= 0.001953125 + 0.017578125 + 0.017578125 + 0.001953125  \\
       &= 0.0390625
     \end{align*}

\end{frame}

%%%%%%
\begin{frame}{Reminder: binomial probabilities}

  \begin{align*}
    &\mbox{Prob}( k \; \text{heads in} \; n \; \text{coin flips} ) \\
    \qquad \qquad &= \ch{n}{k} \; 0.5^k \; 0.5^{n-k} \\
    \qquad \qquad &= \text{(\# ways to get $k$ heads)} \text{(prob of each possibility)}
  \end{align*}

    \vspace{2em}

    \structure{live simulation example}


\end{frame}

%%%%%%
\begin{frame}{What to do about zeros?}

  Again, the null hypothesis is \\
  \hspace{2em} $H_0$: difference between the observations is equally likely to be positive or negative.

    \vspace{2em}

    \structure{So:} got some ties?  (i.e.\ $D=0$) \\
    \alert{Ignore them;} compute the $P$-value as if they weren't there.

    \vspace{2em}

    \structure{Because:} we test the hypothesis that \alert{if the paired observations are different}, $Y_1>Y_2$ is just as likely as $Y_2>Y_1$.  (and so, chance of each is 1/2).

\end{frame}

%%%%%%
\begin{frame}{Example}

\end{frame}

\subsection{Extending the idea}

%%%%%%
\begin{frame}{Using the binomial formula}

  Yeast ({\it Saccaromyces cerevisiae}) can be either haploid (one copy of each chromosome per cell) or diploid (two copies, like us).

\end{frame}
%% TPS: p-value with  Pr{D>0} = 1/3
% Wilcoxon signed-rank test
%% how to do it: compare sums of ranks of + and -
%% conditions: continuous distribution
%% zeros: toss 'em.
%% ties: average 'em.
%% null hypothesis: H0: \mu_D = 0
% limitations to paired sample tests
%% when it's not just about the mean difference
%% always report within-population variation

%%%%%% %%%%%%%%% %%%%%%%%%%
\subsection{Comparison of the tests}

%%%%%%
\begin{frame}{Conditions}

  The sign test versus the Wilcoxon signed-rank test:
  \begin{enumerate}
    \item Both are distribution--free
    \item The sign test is \alert{more flexible} than the Wilcoxon signed-rank test
  \end{enumerate}

    \vspace{2em}

    Sign test:
    \[ H_0: \quad \mbox{Prob}(D>0) = 1/2 . \]

    \vspace{2em}

    Wilcoxon signed-rank test:
    \[ H_0: \quad \text{ distribution of $D$ is symmetric } .\]

\end{frame}

\section<article>{Summary}
\section<presentation>*{Summary}

\begin{frame}{Summary}
  \begin{enumerate}
  \item Ein \alert{Wort} ist eine Folge von Symbolen aus einem
    \alert{Alphabet}. 
  \item Eine \alert{Syntax} besteht aus Regeln, nach denen
    Worte (Texte) gebaut werden d�rfen.
  \item Eine \alert{Semantik} legt fest, was Worte \alert{bedeuten}.
  \item Eine \alert{formale Sprache} ist eine Menge von Worten
    �ber einem Alphabet.
  \end{enumerate}
\end{frame}

% homework
\begin{frame}{Homework}
  \begin{center}

  8.4.3

  \vspace{2em}

  8.4.11

  \vspace{2em}

  8.5.4


  \end{center}
\end{frame}


\end{document}





