% Copyright 2007 by Till Tantau
%
% This file may be distributed and/or modified
%
% 1. under the LaTeX Project Public License and/or
% 2. under the GNU Public License.
%
% See the file doc/licenses/LICENSE for more details.


\lecture[1]{Syntax versus Semantik}{lecture-text}

\subtitle{Text und seine Bedeutung}

\date{17 October 2013}


\begin{document}

\begin{frame}
  \maketitle
\end{frame}


\section*{Ziele und Inhalt}

\begin{frame}{Die Lernziele der heutigen Vorlesung und der �bungen.} 
  \begin{enumerate}
  \item Die Begriffe Syntax und Semantik erkl�ren k�nnen
  \item Syntaktische und semantische Elemente nat�rlicher Sprachen und
    von Programmiersprachen benennen k�nnen
  \item Die Begriffe Alphabet und Wort kennen
  \item Objekte als Worte kodieren k�nnen
  \end{enumerate}
\end{frame}

\begin{frame}\frametitle<presentation>{Gliederung}
  \tableofcontents
\end{frame}

Or, Sticklebacks (has plot): http://www.plosone.org/article/info%3Adoi%2F10.1371%2Fjournal.pone.0059644

From http://www.sciencemag.org.libproxy.usc.edu/content/296/5568/707.full:

The first samples in 1973 (221 G. fortis and 72 G. scandens) were compared by ANOVAs with the last samples in 2001 (114 G. fortis and 35 G. scandens). The data were trimmed to 2.5 SD on either side of the mean by removing one to three individuals from the samples of each species. This corrected for skewness and unequal variances. Sex was included in two-factor ANOVAs because males are generally larger than females. Mean body size was significantly smaller in G. fortis (F 1,132 = 7.773, P = 0.0061) and in G. scandens (F 1,50 = 11.272, P < 0.0001) in 2001 than in 1973. There was a significant effect of sex in each species (P < 0.002) but no sex-by-year interaction (P > 0.1). Mean beak size did not differ between years in either G. fortis (F 1,166 = 0.004, P = 0.9480) or G. scandens (F 1,50 = 3.108, P= 0.0840); sex effects were significant in both species (P< 0.007), but there were no sex-by-year interactions (P> 0.1). Beak shape differed between years in both species. For G. scandens there was a strong year effect (F 1,72 = 17.168, P < 0.0001), a weak sex effect (F 1,72 = 5.943, P = 0.0172), and no interaction. The G. fortis sexes do not differ in beak shape (P = 0.9715), and therefore a one-factor ANOVA was performed with adult males, females, and birds of unknown sex. It demonstrated a strong difference between years (F 1,287 = 30.246, P < 0.0001). 

\section{Was ist Syntax?}

\begin{frame}{Die zwei Hauptbegriffe der heutigen Vorlesung.}
  \begin{block}{Grobe Definition (Syntax)}
    Unter einer \alert{Syntax} verstehen wir \alert{Regeln}, nach denen
    Texte \alert{strukturiert} werden d�rfen. 
  \end{block}
  \begin{block}{Grobe Definition (Semantik)}
    Unter einer \alert{Semantik} verstehen wir die Zuordnung von
    \alert{Bedeutung} zu Text.
  \end{block}
\end{frame}


\subsection[Syntax \protect\\ nat�rlicher Sprachen]{Syntax nat�rlicher Sprachen}

\begin{frame}{Beobachtungen zu einem �gyptischen Text.}
  \includegraphicscopyright[width=6cm]{beamerexample-lecture-pic3.jpg}
  {Copyright by Guillaume Blanchard, GNU Free Documentation License, Low Resultion}

  \begin{block}{Beobachtungen}
    \begin{itemize}
    \item Wir haben keine Ahnung, was der Text bedeutet.
    \item Es gibt aber \alert{Regeln}, die offenbar eingehalten wurden,
      wie �Hieroglyphen stehen in Zeilen�.
    \item Solche Regeln sind \alert{syntaktische Regeln} -- man kann sie
      �berpr�fen, ohne den Inhalt zu verstehen.
    \end{itemize}
  \end{block}
\end{frame}


% . . . 

\section<article>{Summary}
\section<presentation>*{Summary}

\begin{frame}{Summary}
  \begin{enumerate}
  \item Ein \alert{Wort} ist eine Folge von Symbolen aus einem
    \alert{Alphabet}. 
  \item Eine \alert{Syntax} besteht aus Regeln, nach denen
    Worte (Texte) gebaut werden d�rfen.
  \item Eine \alert{Semantik} legt fest, was Worte \alert{bedeuten}.
  \item Eine \alert{formale Sprache} ist eine Menge von Worten
    �ber einem Alphabet.
  \end{enumerate}
\end{frame}

% homework
\begin{frame}{Homework}
  \begin{center}

  % 7.4.2

  \vspace{2em}

  % 7.4.8

  \vspace{2em}

  % 7.5.1

  \end{center}
\end{frame}


\end{document}





