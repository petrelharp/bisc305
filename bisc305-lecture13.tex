\lecture[13]{Association \& Causation}{lecture-text}

\subtitle{and One--sided $t$-tests}

\date{17 October 2013}


\begin{document}

\begin{frame}
  \maketitle
\end{frame}

\begin{frame}{So far we know\ldots}

    \begin{itemize}
        \item How to estimate the mean value of a population from a sample ($\bar y$)
        \item How to quantify uncertainty in estimated mean values ($\SE_{\bar Y}$)
        \item How to compare estimated mean values in two independent samples ($t$-test)
    \end{itemize}

\end{frame}

\section*{Introduction}
\begin{frame}\frametitle<presentation>{Outline}
  \tableofcontents
\end{frame}


\section{Association and Causation}

\begin{frame}{What happens in a study}

    \alert{Question:} Does $X$ affect $Y$?

    \vspace{2em}

    To find out, we get a bunch of samples, recording:
      \begin{enumerate}
          \item explanatory variable(s) ($X$)
          \item response variable(s) ($Y$)
      \end{enumerate}
    and the typical value of $Y$ seems to depend on $X$.

    \vspace{2em}

    \ldots but then, what do we conclude?

    \vspace{2em}

    \alert{It depends} on how we ``get'' those samples.

\end{frame}

\begin{frame}{Example: hematocrit}

    \alert{Hematocrit} measures concentration of red blood cells in blood.

    \vspace{2em}

    \only<1>{
    \begin{block}{Males \& Females}
        From sample of 489 males and 469 females:

        \vspace{1em}

        \centering
        \begin{tabular}{l|rr}
            & Males & Females \\
            \hline
            Mean & 45.8 & 40.6 \\
            SD & 2.8 & 2.9 \\
        \end{tabular}
    \end{block}
    }

    \only<2>{
    \begin{block}{High \& low hormone levels}
        From sample of 489 ``low'' and 469 ``high'' 
        levels of androstenedione:

        \vspace{1em}

        \centering
        \begin{tabular}{l|rr}
            & Low & High \\
            \hline
            Mean & 45.8 & 40.6 \\
            SD & 2.8 & 2.9 \\
        \end{tabular}

        \tiny (fake data)
    \end{block}
    }

    \only<3>{
    \begin{block}{Controlled hormone levels}
        From sample of 489 males and 469 females,
        random set of half of each had
        artificially induced levels of androstenedione:

        \vspace{1em}

        \centering
        \begin{tabular}{l|rr}
            & Low & High \\
            \hline
            Mean & 45.8 & 40.6 \\
            SD & 2.8 & 2.9 \\
        \end{tabular}

        \tiny (fake data)
    \end{block}
    }


    \vspace{2em}

    What do we conclude?

\end{frame}

%%%%%
\subsection{Experimental vs.\ Observational studies}

\begin{frame}{There's more to an experiment than observation.}

  \begin{block}{How to do an observational study}
    \alert{Observe} (measure) some stuff.
  \end{block}

  \vspace{2em}

  \begin{block}{How to do an experimental study}
    Observe (measure) some stuff, 
    in samples from populations that are identical,
    except for the explanatory variable(s).
  \end{block}

  \vspace{2em}

  \alert{Easy way to get experimental:}
  Randomly assign to treatment and control groups.

\end{frame}

%
\begin{frame}{Observation versus experiment}

  Is it an observational study or an experimental study?
  \begin{enumerate}

    \item How was the sample obtained?

    \item Who is in the treatment and the control groups? \\
      \alert{or} Does the researcher assign the explanatory variables?

    \item Who is being studied?

  \end{enumerate}

\end{frame}

%


%
\begin{frame}{The difference}

  In \alert{both} observational and experimental studies,
  the conclusion is generally ``$X$ is associated with $Y$'' (or not).

  \vspace{2em}

  \ldots but in an \alert{experiment},
  there are a lot fewer ways to explain the association.

  \vspace{2em}

  \structure{Example:}
  \begin{description}
      \item[Study 1:] Children of smokers have lower birth weight.
      \item[Study 2:] We got a randomly chosen half of our sample to start smoking,
          and their children had lower birth weight.
  \end{description}

\end{frame}


%%%
\subsection{Correlation is not causation}

%
\begin{frame}{Correlation is not Causation}

  \begin{center}
  \includegraphicscopyright[width=6cm]{xkcd-correlation}{xkcd.com/552}

  \figcaption{``Correlation doesn't imply causation, but it does waggle its eyebrows suggestively and gesture furtively while mouthing `look over there'.''}
  \end{center}

  \vspace{2em}

  \alert{Why not?}

\end{frame}

%%%
\subsection{Confounding factors}

%
\begin{frame}{Confounding factors}

If we are looking for the effect of $X$ on $Y$,
the factor $Z$ is \alert{confounding} if $Z$ is known to have an effect on $Y$,
and $X$ might be associated with $Z$.

\vspace{2em}

If association between $X$ and $Y$ only happens because of $Z$:
\[
  X \sim Z \longrightarrow Y
\]
then the association is \alert{spurious}.

\vspace{2em}

\alert{Solution:} control for $Z$.  (can be difficult)

\vspace{2em}

\structure{Study 2':} We got 159 pairs of women,
    each matching in many ways, except one of the pair quit smoking before the second pregnancy.
    Those who quit had heavier children.

\end{frame}

\begin{frame}{Exercise}

\end{frame}


%%%%
\section{One--sided $t$-tests}

\subsection{One versus two sides}

%
\begin{frame}{One or two sides?}

  \begin{description}

    \item[one-sided test:] ``Could the two groups look \alert{this different} if they were actually the same?''

    \item[two-sided test:] ``Could this group look \alert{this much larger} than the other group if they were actually the same?''

  \end{description}

  \vspace{2em}

  \only<1>{
  \structure{More precisely,}
  \begin{description}

    \item[one-sided test:] ``What's the chance that the difference in sample means was at least this large, if the control and treatment populations actually have the same mean?''

    \item[two-sided test:] ``What's the chance that the treatment sample mean is at least this much larger than the control group sample mean,
      if the control and treatment populations actually have the same mean?''

  \end{description}
  }

  \only<2>{
  \structure{More concisely,} with $H_0: \; \mu_1 = \mu_2$:
  \begin{description}

    \item[one-sided test:] $H_A: \; \mu_1 \neq \mu_2$

    \item[two-sided test:] $H_A: \; \mu_1 > \mu_2$

  \end{description}

  \vspace{2em}

  \structure{Note:} can do the test in either direction, replacing "larger" with "smaller".
  }

\end{frame}

%%%
\subsection{Finding a one-sided $P$-value}

\begin{frame}{How to do a one-sided test}

  \begin{center}
  \includegraphicscopyright[width=.8\textwidth]{one-tailed-tests}{Samuels, Whitmer, \& Schaffner}
  \end{center}

  \vspace{-1em}

  \begin{enumerate}
    \item Check if the difference in means is in the right direction. (if not, $P>0.5$).

    \item Compute the $t$ statistic:
      \[
          t_s = \frac{ (\bar y_1 - \bar y_2) - 0}{ \SE_{(\bar Y_1 - \bar Y_2)} }
      \]

    \item Compute the degrees of freedom:
        \[
            df = \frac{ (\SE_1^2 + \SE_2^2)^2 }{ \SE_1^4 / (n_1-1) + \SE_2^4 / (n_2 - 1) }
        \]

    \item Look up the $P$-value. (or, check for significance in Table 4)

  \end{enumerate}


\end{frame}

%%
\subsection{What to watch out for}

%
\begin{frame}{How to cheat}

  \begin{enumerate}
    \item Check which mean is larger.
    \item Do a one-sided test, in that direction.
  \end{enumerate}

  \vspace{2em}

  \structure{Solution:} Have a clear hypothesis, that determines the direction beforehand.

  \vspace{2em}

  \begin{block}{Music \& Marigolds}
    If 100 studies look for an effect that isn't real, 
    how many will ``find'' the effect, with
    \begin{itemize}
      \item[\bf (a)] two-tailed tests
      \item[\bf (b)] one-tailed tests, and a clear prior hypothesis
      \item[\bf (c)] one-tailed tests, cheating
    \end{itemize}
  \end{block}

\end{frame}


%%%%%
\section<article>{Summary}
\section<presentation>*{Summary}

\begin{frame}{Summary}
  \begin{itemize}
    \item An \alert{observational study} simply observes an existing situation,
    \item and finds correlations.
    \item But, \alert{correlation is not causation},
    \item thanks in part to \alert{confounding factors}.
    \item \alert{Experimental studies} try to eliminate these,
    \item and so can provide much stronger evidence for direct causation.
  \end{itemize}
  \structure{and}
  \begin{itemize}
      \item Use a one-sided $t$-test if the hypothesis clearly predicts a difference in a \alert{particular direction}.
      \item The $P$-value is \alert{just one tail} of the $t$-distribution, not both.
  \end{itemize}
\end{frame}

% homework
\begin{frame}{Homework}

  \begin{center}

  7.4.2

  \vspace{2em}

  7.4.8

  \vspace{2em}

  7.5.1

  \end{center}

\end{frame}



\end{document}





