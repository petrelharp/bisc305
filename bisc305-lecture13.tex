\lecture[13]{Association \& Causation}{lecture-text}

\subtitle{and One--sided $t$-tests}

\date{17 October 2013}


\begin{document}

\begin{frame}
  \maketitle
\end{frame}

\begin{frame}{Overview}

where we're at so far

\end{frame}

\section*{Introduction}
\begin{frame}\frametitle<presentation>{Outline}
  \tableofcontents
\end{frame}


\section{Association and Causation}
\begin{frame}{General experimental set-up}

We observe, in a sample:
  \begin{enumerate}
  \item explanatory variable(s)
  \item response variable(s)
  \end{enumerate}

\end{frame}


\subsection{Experimental vs.\ Observational studies}

\begin{frame}{Experiment $\neq$ Observation}

  \begin{block}{Observational studies}
    We \alert{observe} (measure) some things.
  \end{block}

  \begin{block}{Experimental studies}
    We observe (measure) some stuff in populations that are identical,
    except for the explanatory variable(s).
  \end{block}


\end{frame}


\begin{frame}{Observational studies}
%  \includegraphicscopyright[width=6cm]{beamerexample-lecture-pic3.jpg}

  \begin{block}{Beobachtungen}
    \begin{itemize}
    \item Wir haben keine Ahnung, was der Text bedeutet.
    \item Es gibt aber \alert{Regeln}, die offenbar eingehalten wurden,
      wie »Hieroglyphen stehen in Zeilen«.
    \item Solche Regeln sind \alert{syntaktische Regeln} -- man kann sie
      überprüfen, ohne den Inhalt zu verstehen.
    \end{itemize}
  \end{block}

\end{frame}


% . . . 

\section<article>{Zusammenfassung}
\section<presentation>*{Zusammenfassung}

\begin{frame}{Zusammenfassung}
  \begin{enumerate}
  \item Ein \alert{Wort} ist eine Folge von Symbolen aus einem
    \alert{Alphabet}. 
  \item Eine \alert{Syntax} besteht aus Regeln, nach denen
    Worte (Texte) gebaut werden dürfen.
  \item Eine \alert{Semantik} legt fest, was Worte \alert{bedeuten}.
  \item Eine \alert{formale Sprache} ist eine Menge von Worten
    über einem Alphabet.
  \end{enumerate}
\end{frame}

\end{document}





