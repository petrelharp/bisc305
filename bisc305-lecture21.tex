% Copyright 2007 by Till Tantau
%
% This file may be distributed and/or modified
%
% 1. under the LaTeX Project Public License and/or
% 2. under the GNU Public License.
%
% See the file doc/licenses/LICENSE for more details.


\lecture[21]{Relationships in Categorical Data}{lecture-text}

\subtitle{$\chi^2$ between two samples}

\date{14 November 2013}


% 363-372, 385-389


\begin{document}

\begin{frame}
  \maketitle
\end{frame}


\begin{frame}{The last few classes,}
    working with \alert{categorical data}, we
  \begin{enumerate}
      \item estimated single frequencies,
      \item with confidence intervals,
      \item and did hypothesis tests for equality of frequencies.
  \end{enumerate}
\end{frame}

\begin{frame}\frametitle<presentation>{Outline}
  \tableofcontents
\end{frame}


\section{Multiple categorical samples}

\subsection{Contingency tables}

\begin{frame}{Samples, or more cateogries}

    A \structure{contingency table} gives the \alert{number} of observations
    falling into each pairwise \alert{combination} of two categorizations.

    \vspace{2em}

    \structure{Often,} the entire table represents a single sample.

    \vspace{2em}

    \structure{Sometimes,} rows are \alert{categories} and columns are \alert{samples} from different populations. (e.g.\ males and females; treatment and control)

\end{frame}


%%%%%%
\begin{frame}{Example: HIV}
    A random sample of 120 college students:
    \begin{center}
        \begin{tabular}{lcc|c}
            & female & male & total \\
            \hline
            HIV test & 9 & 8 & 17 \\
            no HIV test & 52 & 51 & 103 \\
            \hline
            total & 61 & 59 & 120 \\
        \end{tabular}
    \end{center}

    \vspace{2em}


\end{frame}

%%%%%%
\begin{frame}{Example: surgery response}
    75 patients were randomly assigned to treatment or placebo:
    \begin{center}
        \begin{tabular}{lcc|c}
            & treatment & placebo & total\\
            \hline
            successful & 8 & 11 & 19 \\
            not successful & 41 & 15 & 56\\
            \hline
            total & 49 & 26 & 75 \\
        \end{tabular}
    \end{center}

    \vspace{2em}

\end{frame}

%%%%%%
\begin{frame}{Example: owl diet}

    \begin{center}
    \includegraphicscopyright[width=\textwidth]{barn-owl-comparison}{Owl predation in Pennsylvania, Pearson \& Pearson 1947}
    \end{center}

    \vspace{2em}


\end{frame}

\subsection{Conditional probability}



%%%%%%
\begin{frame}{An brief excursion}

    A useful concept, and accompanying notation:
    \begin{block}{Conditional Probability}
        The ``\alert{probability of $E$ given $C$}''
        is written:
        \[ \prob\{ E \vert C \} . \]
        One way to think about it: the probability that $E$ happens, among the cases that $C$ happens.
    \end{block}

    \vspace{2em}

    \begin{block}{Conditional Frequencies}
        The empirical fraction of observations for which $E$ and $C$ are both true,
        out of the observations for which $C$ is true,
        is  written
        \[ \hat \prob \{ E \vert C \} .  \]
        It estimates $\prob\{E \vert C\}$.
    \end{block}

\end{frame}


%%%%%%
\begin{frame}{Main point}

    Here, we use conditional probabilities
    to talk about frequencies \alert{in a particular column}
    of a contingency table.

    \vspace{2em}

    \begin{center}
        \begin{tabular}{lcc|c}
            & female & male & total \\
            \hline
            HIV test & 9 & 8 & 17 \\
            no HIV test & 52 & 51 & 103 \\
            \hline
            total & 61 & 59 & 120 \\
        \end{tabular}
    \end{center}

    \vspace{1em}

    \structure{Example:}
    \begin{align*}
        \hat \prob\{ \text{HIV test} \} = \frac{17}{120} \\
        \hat \prob\{ \text{HIV test} \vert \text{female} \} = \frac{9}{61} \\
        \hat \prob\{ \text{HIV test} \vert \text{male} \} = \frac{8}{59} \\
    \end{align*}

\end{frame}

\section{$\chi^2$ on $2\times2$ tables}


%%%%%%
\begin{frame}{$\chi^2$ without theoretical proportions}

    \begin{center}
        \begin{tabular}{lcc|c}
            & female & male & total \\
            \hline
            HIV test & 9 & 8 & 17 \\
            no HIV test & 52 & 51 & 103 \\
            \hline
            total & 61 & 59 & 120 \\
        \end{tabular}
    \end{center}


    \vspace{2em}

    \structure{Question:}
    Are males and females equally likely to be HIV tested?

    \vspace{2em}

    \begin{align*}
        H_0&: \prob\{ \text{HIV test} \vert \text{female} \} = \prob\{ \text{HIV test} \vert \text{male} \} \\
        H_A&: \prob\{ \text{HIV test} \vert \text{female} \} \neq \prob\{ \text{HIV test} \vert \text{male} \}
    \end{align*}

\end{frame}

\subsection{Finding expected counts}

%%%%%%
\begin{frame}{The test statistic}

    \begin{center}
        \begin{tabular}{lcc|c}
            & female & male & total \\
            \hline
            HIV test & 9 & 8 & 17 \\
            no HIV test & 52 & 51 & 103 \\
            \hline
            total & 61 & 59 & 120 \\
        \end{tabular}
    \end{center}

    \vspace{2em}

    \structure{The issue:}
    we don't have theoretical proportions for each category.

    \vspace{2em}
    
    \structure{Solution:}
    estimate them, using \alert{both} categories.

    \vspace{2em}

    \structure{Conceptually,} see if
    \begin{align*}
        \hat \prob\{ \text{HIV test} \vert \text{female} \} 
        \approx \hat \prob\{ \text{HIV test} \vert \text{male} \}
        \approx \hat \prob\{ \text{HIV test} \},
    \end{align*}
    where ``$\approx$'' means ``looks the same up to sampling noise''.


\end{frame}

%%%%%%
\begin{frame}{Expected counts}

    \only<1>{
        \begin{center}
            \begin{tabular}{lcc|c}
                \alert{observed}& female & male & total \\
                \hline
                HIV test & 9 & 8 & 17 \\
                no HIV test & 52 & 51 & 103 \\
                \hline
                total & 61 & 59 & 120 \\
            \end{tabular}
        \end{center}
    }
    \only<2->{
        \begin{center}
            \begin{tabular}{lcc|cc}
                \alert{expected}& female & male & total & \uncover<3->{proportion} \\
                \hline
                HIV test & \uncover<4->{8.64} & \uncover<4->{8.36} & 17 & \uncover<3->{0.142} \\
                no HIV test & \uncover<4->{52.36} & \uncover<4->{50.64} & 103 & \uncover<3->{0.858} \\
                \hline
                total & 61 & 59 & 120 &  \\
            \end{tabular}
        \end{center}
    }

    \vspace{2em}

    \structure{The issue:}
    we don't have theoretical proportions for each category.

    \vspace{2em}
    
    \structure{Solution:}
    estimate them, using \alert{both} categories.

    \vspace{2em}

    \structure{Conceptually,} see if
    \begin{align*}
        \hat \prob\{ \text{HIV test} \vert \text{female} \} 
        \approx \hat \prob\{ \text{HIV test} \vert \text{male} \}
        \approx \hat \prob\{ \text{HIV test} \},
    \end{align*}
    where ``$\approx$'' means ``looks the same up to sampling noise''.


\end{frame}


%%%%%%
\begin{frame}{The test statistic}

    Observed (expected) counts:
        \begin{center}
            \begin{tabular}{lcc|c}
                & female & male & total \\
                \hline
                HIV test & 9 (8.64) & 8 (8.36) & 17 \\
                no HIV test & 52 (52.36) & 51 (50.64) & 103 \\
                \hline
                total & 61 & 59 & 120 \\
            \end{tabular}
        \end{center}

    \vspace{2em}

    The test statistic is the \alert{same} as $\chi^2_s$:
    \[
        \chi^2_s = \frac{(9 - 8.64)^2}{8.64} + 
            \frac{(8 - 8.36)^2}{8.36} + 
            \frac{(52 - 52.36)^2}{52.36} + 
            \frac{(51 - 50.64)^2}{50.64} = 0.035
    \]
    and there is \alert{one} {degree of freedom}, and $P > 0.2$ \\
    so there is no significant evidence that men and women get tested at different rates.


\end{frame}

\section{Larger contingency tables}

%%%%%%
\begin{frame}{General formula:}

    In a general $r \times k$ contingency table:
    \begin{enumerate}
        \item Compute expected counts for each cell by:
            \[ e_i = \frac{ (\text{row total}) \times (\text{column total}) }{ \text{grand total} } . \]
        \item The degrees of freedom are: $df = (r-1)(k-1)$.
        \item The test statistic is:
            \[ \chi^2_s = \sum_i \frac{ ( o_i - e_i )^2 }{ e_i } . \]
    \end{enumerate}

    \vspace{2em}

    The \structure{null hypothesis} is that the (conditional) probabilities of each row category
    do not depend on the column grouping
    (and, vice-versa).

\end{frame}

\subsection{Examples}

%%%%%%
\begin{frame}{Example: plover nests}
    

        \begin{center}
            \begin{tabular}{llll|c}
                 location & 2004 & 2005 & 2006 & total \\
                \hline
                ag field & 21 \uncover<2->{\alert{(18.55)}} & 19 \uncover<2->{\alert{(21.18)}} & 26 \uncover<2->{\alert{(20.27)}} & 66 \\
                prarie dog city & 17 \uncover<2->{\alert{(18.83)}} & 38 \uncover<2->{\alert{(27.59)}} & 12 \uncover<2->{\alert{(20.58)}} & 67 \\
                grassland & 5 \uncover<2->{\alert{(5.62)}} & 6 \uncover<2->{\alert{(8.24)}} & 9 \uncover<2->{\alert{(6.14)}} & 20 \\
                \hline
                total & 43 & 63 & 47 & 153 \\
            \end{tabular}
        \end{center}


    \vspace{2em}

    \noindent
    $H_0$: The population distributions of the nest locations are the same in all three years. \\
    $H_A$: The population distributions of the nest locations differ across years.

    \vspace{2em}

    \uncover<3->{
    \[ \chi^2_s = \frac{(21-18.55)^2}{18.55} +  \cdots + \frac{(9-6.14)^2}{6.14} = 14.09 ,\]
    and $df=(3-1)(3-1)=4$, so $0.001 < P < 0.01$.

    \vspace{1em}
    Strong evidence that nest location preferences \alert{changed} across the years.


    }

\end{frame}


%%%%%%
\begin{frame}{Example: Hair and Eye Color}

    In 6,800 German men, hair (columns) and eye (rows) color were:
        \begin{center}
            \begin{tabular}{lccc|c}
                & brown & black & fair & red \\
                brown & 438 & 288 & 115 & 16 \\
                grey/green & 1387 & 746 & 946 & 53 \\
                blue & 807 & 189 & 1768 & 47 \\
            \end{tabular}
        \end{center}

        \uncover<2->{How does this conclusion differ from the prairie dog example?}
\end{frame}

\section{Conditions}

%%%%%%
\begin{frame}{Conditions}

    The $\chi^2$ test requires that the counts derive from \alert{independent observations}.

    \vspace{2em}

    This is because it uses the Normal approximation to the \structure{Binomial}.
    (think: independent coin flips)


    \vspace{2em}

    \structure{Example:}
    We count the number of fouls made by our team and the opposing team across 10 soccer matches,
    categorize them by type of foul (no card, yellow card, red card),
    and report this in a $2\times 3$ contingency table.
    Can we use $\chi^2$ to test if our team has a different distribution of foul types than its opponents?

\end{frame}


%%%%%%
\begin{frame}{Example:}

    \begin{center}
    \includegraphicscopyright[width=\textwidth]{barn-owl-comparison}{Owl predation in Pennsylvania, Pearson \& Pearson 1947}
    \end{center}


\end{frame}


\section<article>{Summary}
\section<presentation>*{Summary}

\begin{frame}{Summary}
  \begin{enumerate}
      \item A contingency table
      \item provides counts of independent observations
      \item categorized by all pairwise combinations of several categories.
      \item Often, different columns  are different (independent) samples.
      \item A $\chi^2$ test will test for homogeneity of row poportions,
      \item if we compute ``expected'' counts as the product of the marginals.
  \end{enumerate}
\end{frame}

% homework
\begin{frame}{Homework}
  \begin{center}

  10.2.1

  \vspace{2em}

  10.2.5

  \vspace{2em}

  % 7.5.1

  \end{center}
\end{frame}


\end{document}





