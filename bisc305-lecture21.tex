% Copyright 2007 by Till Tantau
%
% This file may be distributed and/or modified
%
% 1. under the LaTeX Project Public License and/or
% 2. under the GNU Public License.
%
% See the file doc/licenses/LICENSE for more details.


\lecture[1]{Syntax versus Semantik}{lecture-text}

\subtitle{Text und seine Bedeutung}

\date{14 November 2013}


% 363-372, 385-389

% Multiple categorical samples
%% contingency tables
%% ex: HIV tests
%% ex: owls & traps
%% questions: same proportions in columns?
% conditional probability
%% population composed of all columns *in different proportions*
%% but conditional is just one column
%% Pr-hat notation
% chi-square with estimated expected (2x2)
%% null hypothesis: populations have same frequencies
%% so estimate expected from both samples
%% with 2x2, df=1: row sums, column sums, total
%% example: HIV
% larger contingency tables
%% H0: homogeneity in column proportions
%% df = (r-1)(k-1)
%% examples


\begin{document}

\begin{frame}
  \maketitle
\end{frame}


\section*{Ziele und Inhalt}

\begin{frame}{Die Lernziele der heutigen Vorlesung und der �bungen.} 
  \begin{enumerate}
  \item Die Begriffe Syntax und Semantik erkl�ren k�nnen
  \item Syntaktische und semantische Elemente nat�rlicher Sprachen und
    von Programmiersprachen benennen k�nnen
  \item Die Begriffe Alphabet und Wort kennen
  \item Objekte als Worte kodieren k�nnen
  \end{enumerate}
\end{frame}

\begin{frame}\frametitle<presentation>{Gliederung}
  \tableofcontents
\end{frame}


\section{Was ist Syntax?}

\begin{frame}{Die zwei Hauptbegriffe der heutigen Vorlesung.}
  \begin{block}{Grobe Definition (Syntax)}
    Unter einer \alert{Syntax} verstehen wir \alert{Regeln}, nach denen
    Texte \alert{strukturiert} werden d�rfen. 
  \end{block}
  \begin{block}{Grobe Definition (Semantik)}
    Unter einer \alert{Semantik} verstehen wir die Zuordnung von
    \alert{Bedeutung} zu Text.
  \end{block}
\end{frame}


\subsection[Syntax \protect\\ nat�rlicher Sprachen]{Syntax nat�rlicher Sprachen}

\begin{frame}{Beobachtungen zu einem �gyptischen Text.}
  \includegraphicscopyright[width=6cm]{beamerexample-lecture-pic3.jpg}
  {Copyright by Guillaume Blanchard, GNU Free Documentation License, Low Resultion}

  \begin{block}{Beobachtungen}
    \begin{itemize}
    \item Wir haben keine Ahnung, was der Text bedeutet.
    \item Es gibt aber \alert{Regeln}, die offenbar eingehalten wurden,
      wie �Hieroglyphen stehen in Zeilen�.
    \item Solche Regeln sind \alert{syntaktische Regeln} -- man kann sie
      �berpr�fen, ohne den Inhalt zu verstehen.
    \end{itemize}
  \end{block}
\end{frame}


% . . . 

\section<article>{Summary}
\section<presentation>*{Summary}

\begin{frame}{Summary}
  \begin{enumerate}
  \item Ein \alert{Wort} ist eine Folge von Symbolen aus einem
    \alert{Alphabet}. 
  \item Eine \alert{Syntax} besteht aus Regeln, nach denen
    Worte (Texte) gebaut werden d�rfen.
  \item Eine \alert{Semantik} legt fest, was Worte \alert{bedeuten}.
  \item Eine \alert{formale Sprache} ist eine Menge von Worten
    �ber einem Alphabet.
  \end{enumerate}
\end{frame}

% homework
\begin{frame}{Homework}
  \begin{center}

  % 7.4.2

  \vspace{2em}

  % 7.4.8

  \vspace{2em}

  % 7.5.1

  \end{center}
\end{frame}


\end{document}





