% Copyright 2007 by Till Tantau
%
% This file may be distributed and/or modified
%
% 1. under the LaTeX Project Public License and/or
% 2. under the GNU Public License.
%
% See the file doc/licenses/LICENSE for more details.


\lecture[21]{Relationships in Categorical Data}{lecture-text}

\subtitle{$\chi^2$ between two samples}

\date{14 November 2013}


% 363-372, 385-389


\begin{document}

\begin{frame}
  \maketitle
\end{frame}


\begin{frame}{The last few classes,}
    working with \alert{categorical data}, we
  \begin{enumerate}
      \item estimated single frequencies,
      \item with confidence intervals.
      \item did hypothesis tests for equality of frequencies
  \end{enumerate}
\end{frame}

\begin{frame}\frametitle<presentation>{Outline}
  \tableofcontents
\end{frame}


\section{Multiple categorical samples}

\subsection{Contingency tables}

\begin{frame}{Samples, or more cateogries}

    A \structure{contingency table} gives the \alert{number} of observations
    falling into each pairwise \alert{combination} of two categorizations.

    \vspace{2em}

    \structure{Usually,} rows are \alert{categories} and columns are \alert{samples} from different populations.

    \vspace{2em}

    \structure{The key} is to think about which combinations of numbers can be viewed as a sample from a population.

\end{frame}


%%%%%%
\begin{frame}{Example: HIV}
    A random sample of 120 college students:
    \begin{center}
        \begin{tabular}{lcc}
            & female & male \\
            \hline
            HIV test & 9 & 8 & \\
            no HIV test & 52 & 51 \\
        \end{tabular}
    \end{center}

    \vspace{2em}

    All 120 is a single sample,\\
    but unless we're interested in the sex ratios at this college\\
    we treat each \alert{column} as an independent sample.

\end{frame}

%%%%%%
\begin{frame}{Example: surgery response}
    75 patients were assigned to treatment or placebo
    \begin{center}
        \begin{tabular}{lcc}
            & treatment & placebo \\
            \hline
            successful & 8 & 11 & \\
            not successful & 41 & 15 \\
        \end{tabular}
    \end{center}

    \vspace{2em}

    Here each column is an independent sample.

\end{frame}

%%%%%%
\begin{frame}{Recall the owls}

    \begin{center}
    \includegraphicscopyright[width=\textwidth]{barn-owl-comparison}{Owl predation in Pennsylvania, Pearson \& Pearson 1947}
    \end{center}

    \vspace{2em}

    What we {really} wanted to do here was see if the \alert{two columns} had the same sets of proportions.

\end{frame}

\subsection{Conditional probability}



% conditional probability
%% population composed of all columns *in different proportions*
%% but conditional is just one column
%% Pr-hat notation
% chi-square with estimated expected (2x2)
%% null hypothesis: populations have same frequencies
%% so estimate expected from both samples
%% with 2x2, df=1: row sums, column sums, total
%% example: HIV
% larger contingency tables
%% H0: homogeneity in column proportions
%% df = (r-1)(k-1)
%% examples


% . . . 

\section<article>{Summary}
\section<presentation>*{Summary}

\begin{frame}{Summary}
  \begin{enumerate}
  \item Ein \alert{Wort} ist eine Folge von Symbolen aus einem
    \alert{Alphabet}. 
  \item Eine \alert{Syntax} besteht aus Regeln, nach denen
    Worte (Texte) gebaut werden d�rfen.
  \item Eine \alert{Semantik} legt fest, was Worte \alert{bedeuten}.
  \item Eine \alert{formale Sprache} ist eine Menge von Worten
    �ber einem Alphabet.
  \end{enumerate}
\end{frame}

% homework
\begin{frame}{Homework}
  \begin{center}

  % 7.4.2

  \vspace{2em}

  % 7.4.8

  \vspace{2em}

  % 7.5.1

  \end{center}
\end{frame}


\end{document}





