% Copyright 2007 by Till Tantau
%
% This file may be distributed and/or modified
%
% 1. under the LaTeX Project Public License and/or
% 2. under the GNU Public License.
%
% See the file doc/licenses/LICENSE for more details.


\lecture[17]{the Paired sample $t$}{lecture-text}

\subtitle{and confidence intervals}

\date{31 October 2013}


% pp 299-309


\begin{document}

\begin{frame}
  \maketitle
\end{frame}


\begin{frame}{So far we can}
  \begin{enumerate}
  \item Estimate the mean value in a population from a sample, and
  \item estimate the difference in population means based on samples from each,
  \item in both cases assessing uncertainty by constructing a confidence interval from the $t$ distribution.
  \item We can also use the Wilcoxon--Mann--Whitney test, which has fewer assumptions.
  \item But, these all assume \alert{independent samples}.
  \end{enumerate}
\end{frame}

\begin{frame}\frametitle<presentation>{Outline}
  \tableofcontents
\end{frame}

%%%%%% %%%%%%% %%%%%%%%
\section{Paired sample tests}

%%%%%% %%%%%%% 
\subsection{Paired samples}

%%%%%%
\begin{frame}{Samples, in pairs}

  \begin{block}{Paired samples}
    are two samples, with a one-to-one relationship between observations in each.

    \vspace{1em}

    \structure{Equivalently,} it is a \alert{single} sample, of \alert{pairs of observations}.
  \end{block}

    \vspace{2em}

    As before, we'll need each pair of observations to be independent of the other pairs.

\end{frame}


%%%%%%
\begin{frame}{Example: before and after}

  \structure{Often,} these are measurements on the same individual, before and after some treatment.

    \vspace{1em}

    \structure{Example:} Blood flow, measured before and after caffeine consumption.

    \vspace{1em}
    
    \structure{Or,} comparisons between matched pairs.

    \vspace{1em}
    
    \structure{Example:} twin studies.

    \vspace{1em}
    
    \structure{Example:} birth weight for women, matched for first birth weight and other variables, one of whom quit smoking.

\end{frame}

%%%%%%
\begin{frame}{Advantages to the paired design}

  Having a paired design often
  \begin{enumerate}
    \item controls for potential confounding factors, and
    \item increases power by accounting for within-population variability.
  \end{enumerate}

    \vspace{2em}

    \structure{Example:} Suppose that people's feet are on average 20cm long, with a standard deviation of 3cm; but that right feet are reliably 0.1cm longer than the left feet, with a standard deviation of .01cm.  Will we be able to see this effect with 100 people?
    % Unpaired: 3/10 = 0.6 => No!
    % Paired: 0.1/10 = .01 => Yes!

\end{frame}

%%%%%%
\begin{frame}{The difference}

    \structure{Before,} we analyzed differences in sample means.

    \vspace{2em}

    \structure{With paired samples,} we analyze the mean of the differences.

\end{frame}

%%%%%%
\begin{frame}{Example from the wild}

  \structure{What was the experimental design?}
    \begin{quote} \small
      The human circadian pacemaker controls the timing of the release of the pineal
      hormone melatonin, which promotes sleep, decreases body temperature, and
      diminishes cognitive performance. Although melatonin secretion
      is directly suppressed by exposure to light in a nonlinear intensity-dependent
      fashion, little research has focused on the effect of prior photic history on
      this response. We examined eight subjects in controlled laboratory conditions
      using a within-subjects design. Baseline melatonin secretion was monitored
      %under constant routine conditions and compared with two additional constant routines  with
      [during]
      a fixed light stimulus for 6.5 h of 200 lux after
      approximately 3 d of photic exposure during the subjective day of either about
      200 lux or about 0.5 lux. We found a
      significant increase in melatonin suppression during the stimulus after a prior
      photic history of approximately 0.5 lux compared with approximately 200 lux,
      revealing that humans exhibit adaptation of circadian photoreception. 
      % Such adaptation indicates that translation of a photic stimulus into drive on the human circadian pacemaker involves more complex temporal dynamics than previously recognized. Further elucidation of these properties could prove useful in potentiating light therapies for circadian and affective disorders.
    \end{quote}


\end{frame}


%%%%%% %%%%%%
\subsection{Mean differences}

%%%%%%
\begin{frame}{Notation}

  Similar to before,
  \begin{itemizew}{2.5em}
    \item[$Y_1$] is an observed sample from population 1,
    \item[$Y_2$] is the \alert{corresponding} observation from population 2,
    \item[$D=Y_1-Y_2$] is their difference.
    \item[$\mu_D$] is the population mean of the differences
  \end{itemizew}
  and
  \begin{itemize}
    \item[$\bar d$] is the sample mean of the difference 
    \item[$s_D$] is the sample standard deviation of the differences
    \item[$n_D$] is the number of pairs of samples
    \item[$\SE_{\bar D}$] is the standard error of $\bar D$
  \end{itemize}

    \vspace{2em}

\end{frame}

%%%%%%
\begin{frame}{Back to single-sample statistics}

    The differences are independent observations, and so:
    \begin{align*}
      \SE_{\bar D} = \frac{s_D}{\sqrt{n_D}} .
    \end{align*}

    \vspace{2em}

    Often, the null hypothesis is:
    \[ H_0: \; \mu_D = 0 .\]
    and the alternative hypothesis is
    \[ H_A: \; \mu_D \neq 0 .\]

\end{frame}

%%%%% %%%%%
\subsection{Confidence intervals}

%%%%%%
\begin{frame}{Confidence intervals}

  \structure{Main point:} The differences between pairs are effecitvely a \alert{single sample} of differences.

    \vspace{2em}

    Finding confidence intervals is \alert{the same} as \structure{single-sample $t$}, with slightly different notation.

    \vspace{2em}

    A \alert{95\% confidence interval} for $\mu_D$ is
    \[  \bar d \pm t_{n_D-1,0.025} \SE_{\bar D} ,\]

    \vspace{2em}

    i.e.\ determined from Student's $t$ distribution with $n_D-1$ degrees of freedom.

\end{frame}

%%%%%%
\begin{frame}{Example}

  Myocardial blood flow, eight subjects, before (baseline, $y_1$) and after ($y_2$) caffeine consumption
  \begin{center}
    \begin{tabular}{lrrr}
      \hline
      subject & baseline & caffeine & difference \\ 
      \hline
      1 & 6.37 & 4.52 & 1.85 \\ 
      2 & 5.69 & 5.44 & 0.25 \\ 
      3 & 5.58 & 4.70 & 0.88 \\ 
      4 & 5.27 & 3.81 & 1.46 \\ 
      5 & 5.11 & 4.06 & 1.05 \\ 
      6 & 4.89 & 3.22 & 1.67 \\ 
      7 & 4.70 & 2.96 & 1.74 \\ 
      8 & 3.53 & 3.20 & 0.33 \\ 
      \hline
      Mean & 5.14 & 3.99 & 1.15 \\ 
       SD & 0.83 & 0.86 & 0.63 \\ 
     \end{tabular}
  \end{center}

\end{frame}


%%%%%% %%%%%%
\subsection{Ignoring pairing}
%%%%%%
\begin{frame}{Pairs are good}

    \structure{Example:} hunger rating with and without mCPP.
    \begin{center}
      \begin{tabular}{lrrr}
        \hline
        subject & mCPP & placebo & difference \\ 
        \hline
        1 & 79 & 78 & 1 \\ 
        2 & 48 & 54 & -6 \\ 
        3 & 52 & 142 & -90 \\ 
        4 & 15 & 25 & -10 \\ 
        5 & 61 & 101 & -40 \\ 
        6 & 107 & 99 & 8 \\ 
        7 & 77 & 94 & -17 \\ 
        8 & 54 & 107 & -53 \\ 
        9 & 5 & 64 & -59 \\ 
        \hline
         Mean & 55 & 85 & -30 \\ 
         SD & 32 & 34 & 33 \\ 
         \hline
      \end{tabular}
    \end{center}
\end{frame}

%%%%%%
\begin{frame}{Pairs are good}

    \structure{Example:} hunger rating with and without mCPP.
    \begin{center}
      \begin{tabular}{lrrr}
        \hline
        subject & mCPP & placebo & difference \\ 
        \hline
        \vdots & \vdots & \vdots & \vdots \\
        \hline
         Mean & 55 & 85 & -30 \\ 
         SD & 32 & 34 & 33 \\ 
         \hline
      \end{tabular}
    \end{center}

    \begin{columns}
      \begin{column}{.5\textwidth}
        \structure{Unpaired}:
      \begin{align*} 
        \SE_{(\bar Y_1 - \bar Y_2)} &= \sqrt{\frac{32^2}{9} + \frac{{34^2}}{9}} \\ 
        &= 15.6 
        \end{align*}
        and so
        \begin{align*}
      t_s &= \frac{55-85}{15.6} = -1.92  \\
      \text{\alert{wrongly,}} \; P&=0.075
    \end{align*}

      \end{column}
      \begin{column}{.5\textwidth}
        \structure{Paired}
        \begin{align*}
          \SE_{\bar D} = \frac{33}{\sqrt{9}} = 11
        \end{align*}
        and so
        \[  t_s = \frac{-30-0}{11} = -2.72 \]
        and $P = 0.027$.

      \end{column}
    \end{columns}

    \centering
    \alert{Why, intuitively?}

\end{frame}

%%%%%%
\begin{frame}{From the wild}

  \structure{Example:} (Boomsma et al 2002, NRG)
  \begin{quote}
Studies of the effects of fetal and infant growth on later health in twins have looked at the differences in birth weight in monozygotic and dizygotic twins and at their association with differences in cardiovascular and metabolic parameters. Twin studies can resolve whether these associations are causal, or due to shared genetic factors. For blood pressure, it turned out that the association between low birth weight and high blood pressure in later life is mediated by common genes.
  \end{quote}

    \vspace{2em}

    How would you conduct such a study?

\end{frame}


%%% exercise

\section<article>{Summary}
\section<presentation>*{Summary}

\begin{frame}{Summary}
  \begin{enumerate}
    \item Paired samples can control for confounding variables and other sources of variability.
    \item Pairs should be \alert{independent} of each other.
    \item When analyzing paired samples, treat the differences as a single sample from a population.
  \end{enumerate}
\end{frame}

% homework
\begin{frame}{Homework}
  \begin{center}

  8.2.2

  \vspace{2em}

  8.2.3


  \end{center}
\end{frame}


\end{document}





